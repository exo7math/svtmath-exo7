\documentclass[11pt,class=report,crop=false]{standalone}
\usepackage{exo7sv}

\begin{document}

%%%%%%%%%%%%%%%%%%%%%%%%%%%%%%%%%%%%%%%%%%%%%%%%%%%%%%%%%%%%%%%%%%%%%%
%%%%%%%%%%%%%%%%%%%%%%%%%%%%%%%%%%%%%%%%%%%%%%%%%%%%%%%%%%%%%%%%%%%%%%

\entete{Université de Lille}{Mathématiques pour la SVT}

\titre{Fiche 6. \quad \'Equations différentielles 2} 

\encadre{
	\emph{Savoir.}
	\begin{itemize}[label=$\square$]
		\item Comprendre ce qu'est une condition initiale d'une équation différentielle.
		\item Savoir qu'un condition initiale entraîne l'unicité de la solution.
        \item Savoir interpréter l'unicité en terme des courbes solutions qui ne s'intersectent pas.
		
	\end{itemize}
	\emph{Savoir-faire.}
	\begin{itemize}[label=$\square$]
		\item Savoir résoudre une équation différentielle avec condition initiale.
		\item Savoir obtenir des informations sur une solution sous la condition que les courbes des solutions ne s'intersectent pas.
	\end{itemize}
}

\insertvideo{bOMW8PyixyE}{Fiche 6. \'Equations différentielles (fin)}

\setcounter{equation}{0}

%%%%%%%%%%%%%%%%%%%%%%%%%%%%%%%%%%%%%%%%%%%%%%%%%%%
\subsection*{Un exemple}

Une équation différentielle a en général une infinité de fonctions solutions.

Considérons par exemple l'équation différentielle :
\begin{equation}
y'(x) = y(x)
\label{eq:eqdiff1}
\end{equation}

Alors les solutions de (\ref{eq:eqdiff1}) sont les fonctions :
$$y(x) = ke^x \qquad \text{ où } k \in \Rr.$$ 
Ainsi, chaque valeur de la constante $k$ fournit une fonction solution :
par exemple pour $k=1$, $y_1(x) = e^x$ est solution, pour $k=-2$, 
$y_{-2}(x) = -2e^x$ est solution, pour $k=0$, $y_0(x)=0$ est solution\ldots


Pour n'avoir qu'une seule solution, il faut imposer une condition initiale
\begin{equation}
\left\{\begin{array}{lr}
y'(x) = y(x) &\qquad \text{équation différentielle}\\
y(0) = 3    &\qquad \text{condition initiale}
\end{array}\right.
\label{eq:eqdiff2}
\end{equation}

Les solutions de l'équation différentielle sont de la forme $y(x)=ke^x$,
mais on veut $y(0)=3$. 
Comme $y(0)=ke^0=k$, on doit avoir $k=3$. 
Ainsi l'unique solution du problème (\ref{eq:eqdiff2}) est la fonction 
$y(x) = 3e^x$.

\emph{Exercice.}
Considérons l'équation différentielle avec condition initiale :
\begin{equation}
\left\{\begin{array}{l}
y'(x) = y(x)\\
y(1) = 2    
\end{array}\right.
\label{eq:eqdiff3}
\end{equation}
Trouver l'unique solution de ce problème. (Attention ce n'est pas $y(x) = 2e^x$ !)

%%%%%%%%%%%%%%%%%%%%%%%%%%%%%%%%%%%%%%%%%%%%%%%%%%%
\subsection*{Condition initiale}

\subsubsection*{Définition}
Pour une équation différentielle dite d'ordre $1$, faisant intervenir $y$ et $y'$, une \textbf{condition initiale} est du type :
$$y(x_0)=y_0$$
où $x_0\in\Rr$ et $y_0\in\Rr$ sont des constantes.

\emph{Exemple.}
$$y'(x)=-y(x)^2+3y(x)-2 \quad\text{ et }\quad y(0)=10$$


Le système formé par une équation différentielle $(E)$ et une condition initiale est appelé \textbf{problème de Cauchy}.

Pour une équation différentielle dite d'ordre $2$, faisant intervenir $y$, $y'$ et $y''$, une \textbf{condition initiale} est du type :
$$
\left\{\begin{array}{l}
y(x_0)=y_0\\
y'(x_0)=y_1
\end{array}\right.$$

\emph{Exemple.}
$$y''(x)=-y'(x)+2y(x) \quad\text{ avec }\quad y(0)=0 \ \text{ et }\ y'(0)=1$$

\subsubsection*{Théorème d'unicité}
Pour les problème que l'on rencontrera on admettra le théorème de Cauchy : 
\mybox{\og Une équation différentielle avec condition initiale admet une unique solution.\fg}


%%%%%%%%%%%%%%%%%%%%%%%%%%%%%%%%%%%%%%%%%%%%%%%%%%%
\subsection*{Courbes solutions}

Une \textbf{courbe solution} d'une équation différentielle $(E)$
est le graphe d'une solution de $(E)$.


Pour l'équation différentielle 
$$y'(x) = y(x)$$
on sait que les solution sont les $y(x) = ke^x$, où $k\in \Rr$ est une constante. Ci-dessous sont tracés quelques graphes de ces solutions.

\begin{center}
\begin{tikzpicture}[scale=0.8]

  \draw[->,>=latex,thick,gray] (-6.5,0) -- (2.4,0) node[below left,black] {$x$};
  \draw[->,>=latex,thick,gray] (0,-5) -- (0,5) node[left,black] {$y$};

\begin{scope}[xscale=1]
\foreach \k in {-3,-2.5,...,3} {
  \draw[thick, color=myred,domain=-6:2, smooth,samples=50] plot (\x,{\k*exp(+0.33*\x)});
}
\end{scope}

%\node[blue] at (-3,3) {Cas \  $a>0$};

\draw[blue] (2.3,6)--(2.5,6)--(2.5,0.2)--(2.3,0.2);
\draw[blue] (2.3,-6)--(2.5,-6)--(2.5,-0.2)--(2.3,-0.2);
\node[blue, right] at (3,3) {$k>0$};
\node[blue, right] at (3,0) {$k=0$};
\node[blue, right] at (3,-3) {$k<0$};

\end{tikzpicture}
\end{center}


Le théorème de Cauchy pour les équations
différentielles linéaires se reformule ainsi :
\mybox{
\og Par chaque point $(x_0,y_0) \in \Rr^2$
passe une et une seule courbe solution. \fg}

En particulier :
\mybox{
\og 
Deux courbes solutions ne s'intersectent pas.
\fg}


\emph{Exemple.}
Les solutions de l'équation différentielle
$y'+y=x$ sont les
$$y(x) = x-1 + ke^{-x} \quad k \in\Rr.$$

Pour chaque point $(x_0,y_0) \in \Rr^2$, il existe une unique solution
$y$ telle que $y(x_0)=y_0$. Le graphe de cette solution
est la courbe intégrale passant par $(x_0,y_0)$.

\begin{center}
\begin{tikzpicture}

  \draw[->,>=latex,thick,gray] (-3.5,0) -- (6.5,0) node[below,black] {$x$};
  \draw[->,>=latex,thick,gray] (0,-3.5) -- (0,5) node[left,black] {$y$};
\begin{scope}
    \clip (-3,-3) rectangle (6,4.5);
\begin{scope}[xscale=1.5]

\foreach \k in {-1,-.75,...,4} {
  \draw[thick, color=myred,domain=-2:4, smooth,samples=10] plot (\x,{\x-1+\k*exp(-0.33*\x)});
}
\foreach \k in {4,4.25,...,7} {
  \draw[thick, color=myred,domain=-1.57:4, smooth,samples=10] plot (\x,{\x-1+\k*exp(-0.33*\x)});
}
\foreach \k in {-1.25,-1.5,...,-4} {
  \draw[thick, color=myred,domain=-1:4, smooth,samples=10] plot (\x,{\x-1+\k*exp(-0.33*\x)});
}
\foreach \k in {-4.25,-4.5,...,-8} {
  \draw[thick, color=myred,domain=1:4, smooth,samples=10] plot (\x,{\x-1+\k*exp(-0.33*\x)});
}
\foreach \k in {-8.25,-8.5,...,-12} {
  \draw[thick, color=myred,domain=2:4, smooth,samples=10] plot (\x,{\x-1+\k*exp(-0.33*\x)});
}
\foreach \k in {-12.25,-12.5,...,-20} {
  \draw[thick, color=myred,domain=2.5:4, smooth,samples=8] plot (\x,{\x-1+\k*exp(-0.33*\x)});
}
\foreach \k in {-20.25,-20.5,...,-23} {
  \draw[thick, color=myred,domain=3.7:4, smooth,samples=5] plot (\x,{\x-1+\k*exp(-0.33*\x)});
}

\def\k{3}
\draw[ultra thick, color=myred,domain=-2:4, smooth,samples=10] plot (\x,{\x-1+\k*exp(-0.33*\x)});
\end{scope}
\end{scope}

\fill[blue] (-1,2.07)  circle (2pt) node [below] {$(x_0,y_0)$}; 

\draw (-3,-3) rectangle (6,4.5);
\end{tikzpicture}
\end{center}


%%%%%%%%%%%%%%%%%%%%%%%%%%%%%%%%%%%%%%%%%%%%%%%%%%%
\subsection*{Variations des solutions}

L'équation différentielle permet parfois d'avoir des informations sur la fonctions $f$ avant même de connaître exactement la solution.

Par exemple, considérons :
$$y'(x) = y^2(x) + 1$$
Alors, sans résoudre cette équation, on sait qu'une solution vérifiera $y'(x)\ge0$, donc une solution $y(x)$ est une fonction croissante.

Autre exemple, avec :
$$y'(x) = xe^{y(x)}$$
Si $x\le0$ alors $xe^{y(x)}\le0$ donc une solution vérifiera $y'(x)\le0$, et la solution $y(x)$ est une fonction décroissante sur $]-\infty,0]$. Par contre 
Si $x\ge0$ alors $xe^{y(x)}\ge0$ donc une solution vérifiera $y'(x)\ge0$, et la fonction $y(x)$ est croissante sur $[0,+\infty[$.


%%%%%%%%%%%%%%%%%%%%%%%%%%%%%%%%%%%%%%%%%%%%%%%%%%%
\subsection*{\'Equation à variables séparées}


Une \textbf{équation différentielle à variable séparées} est de la forme :
$$y'(x) = \frac{a(x)}{b(y)}$$
Ce nom est justifié car on peut mettre tous les $x$ d'un côté et tous les $y$ de l'autre :
$$b(y) y' = a(x)$$

Voici la méthode pour résoudre ce type d'équation.


\emph{Exemple.}
$$x^2y'(x) = e^{-y(x)}$$
\begin{itemize}
  \item On sépare les variables $x$ des variables $y$ :
$$y'(x)e^{y(x)} = \frac{1}{x^2}$$
  \item On intègre chacun des côtés :
  $$e^{y(x)} = -\frac1x + c$$
  où $c\in\Rr$ est une constante.
  \item On exprime la solution :
  $$y(x) = \ln\left(-\frac1x + c\right)$$
\end{itemize}

\emph{Autres exemples.} 
$$y'y^2 = x \qquad y'=y\ln(x) \qquad y'=\frac{1}{y^n}$$


\end{document}