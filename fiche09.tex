\documentclass[11pt,class=report,crop=false]{standalone}
\usepackage{exo7sv}

\begin{document}

%%%%%%%%%%%%%%%%%%%%%%%%%%%%%%%%%%%%%%%%%%%%%%%%%%%%%%%%%%%%%%%%%%%%%%
%%%%%%%%%%%%%%%%%%%%%%%%%%%%%%%%%%%%%%%%%%%%%%%%%%%%%%%%%%%%%%%%%%%%%%

\entete{Université de Lille}{Mathématiques pour la SVT}

\titre{Fiche 9. \quad Changement de variable} 

\encadre{
	\emph{Savoir.}
	\begin{itemize}[label=$\square$]
		\item Comprendre la formule du changement de variable.
	\end{itemize}
	\emph{Savoir-faire.}
	\begin{itemize}[label=$\square$]
		\item Savoir trouver le bon changement de variable.
        \item Savoir changer les bornes de l'intégrale.
        \item Savoir changer l'élément différentiel.
	\end{itemize}
}

\insertvideo{n09CCoqDyFU}{Fiche 9. Changement de variable}

\bigskip

%%%%%%%%%%%%%%%%%%%%%%%%%%%%%%%%%%%%%%%%%%%%%%%%%%%%%%%%%%%%%%%%%%%%%%
\subsection*{Formule du changement de variable}



Calculer l'intégrale $\int_a^b  f(u(x)) u'(x) \, \dd x$ par la formule du changement de variable
c'est utiliser la formule suivante :

\mybox{$\displaystyle \int_{a}^{b} f(u(x)) \, u'(x)\,\dd x
=\int_{u(a)}^{u(b)} f(u)\,\dd u$}

Pour la pratique du changement de variable :
\begin{enumerate}
  \item On doit identifier quelle est la fonction $u(x)$ qui réalise la bon changement de variable.
  Ce n'est pas toujours évident, il faut parfois faire plusieurs essais. Le but est d'obtenir une intégrale (à droite dans la formule ci-dessus) plus simple à calculer.

  \item Il faut calculer le nouvel élément différentiel : $\dd u = u'(x) \, \dd x$.

  \item Il faut changer les bornes de l'intégrale : si $x$ varie de $a$ à $b$, alors $u$ varie de $u(a)$ à $u(b)$.
  \item L'intégrale à calculer (en la variable $x$) est remplacée par une intégrale en la variable $u$ (normalement plus simple).
\end{enumerate}

Cela peut sembler un peu compliqué, mais après avoir pratiqué plusieurs exemples c'est assez facile !


%%%%%%%%%%%%%%%%%%%%%%%%%%%%%%%%%%%%%%%%%%%%%%%%%%%%%%%%%%%%%%%%%%%%%%
\subsection*{Exemples} 


\textbf{Exemple 1.}

On veut calculer \[\int_{0}^{\frac{\pi}{6}}\cos(3x)\dd x.\]
On pose le changement de variable $u(x)=3x$. On a alors $u'(x)=3$, donc $\dd u = 3 \,\dd x$, ou encore
$\dd x = \frac{\dd u}{3}$.
Comme $x$ varie de $x=0$ à $x=\frac\pi6$, alors $u$ varie de $u=0$ à $u=\frac\pi2$.
Ainsi,  
$$
\int_{x=0}^{x=\frac{\pi}{6}}\cos(3x)\,\dd x 
= \int_{u=0}^{u=\frac{\pi}{2}} \cos(u) \, \frac{\dd u}{3}
$$

L'intégrale de droite est quand même plus facile à calculer !

Concluons :
$$
\int_{x=0}^{x=\frac{\pi}{6}}\cos(3x)\,\dd x 
= \int_{u=0}^{u=\frac{\pi}{2}} \cos(u) \, \frac{\dd u}{3}
= \frac13 \big[\sin(x)\big]_{0}^{\frac{\pi}{2}}
=\frac{1}{3}\big(\sin(\tfrac{\pi}{2})-\sin(0)\big)
=\frac{1}{3}.$$

\bigskip

\textbf{Exemple 2.}

On veut calculer \[\int_{0}^{1}\frac{e^x}{\sqrt{e^x+1}}\dd x.\]
On pose le changement de variable $u(x)=e^x$. On a alors $u'(x)=e^x$, donc $\dd u = e^x \,\dd x$.
Comme $x$ varie de $x=0$ à $x=1$, alors $u$ varie de $u=e^0=1$ à $u=e^1=e$.
Ainsi la formule du changement de variable donne : 
\[\int_{x=0}^{x=1}\frac{e^x \,\dd x}{\sqrt{e^x+1}}=\int_{u=1}^{u=e} \frac{\dd u}{\sqrt{u+1}}.\]

La seconde intégrale, avec la variable $u$, est plus simple à calculer car une primitive de $\frac{1}{\sqrt{u+1}}$ est $2\sqrt{u+1}$. Ainsi :

\[
\int_{x=0}^{x=1}\frac{e^x \,\dd x}{\sqrt{e^x+1}}
= \int_{u=1}^{u=e} \frac{\dd u}{\sqrt{u+1}}\dd x
= \left[ 2\sqrt{u+1} \right]_{u=1}^{u=e} 
= 2\big( \sqrt{e+1}-\sqrt{2} \big).
\]



\end{document}