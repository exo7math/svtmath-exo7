   % ----------------------------------------------------------------
%
%   Programme SVTE
%
% ----------------------------------------------------------------

% ~~~~~~~~~~~~~~~~~~~~~~~~~~~~~~~~~~~~~~~~~~~~~~~~~
\likesection{Programme}
% ~~~~~~~~~~~~~~~~~~~~~~~~~~~~~~~~~~~~~~~~~~~~~~~~~

% =================================================
\prsec{Calcul différentiel}{15}
% =================================================

    \prsubsec{Rappels sur les fonctions usuelles~:} domaine de définition, graphe, dérivée, limites (fonctions affines, polynomiales, racine, inverse, exp, ln, puissance, sin, cos, tan).

    \prsubsec{Dérivée~:} règles de calculs (à partir des dérivées des fonctions usuelles) : somme, produit, quotient, composition de fonction.


    \prsubsec{Limites~:}
    \begin{itemize}
        \item règles de calcul et formes indéterminées, règle de l'Hospital;

        \item une fonction monotone bornée admet une limite en $\pm \infty$.
    \end{itemize}

    \prsubsec{Études de fonctions~:} domaine de définition, tableaux de variations, extrema, limites aux bords.

    \prsubsec{Équation différentielles~:}
    \begin{itemize}
        \item explication des termes d'une équation différentielle à partir des notions comme \enquote{effectif}, \enquote{taux de croissance}, \enquote{proportionnalité};

        \item vérification qu'une fonction est solution d'une équation;

        \item recherche des solutions constantes;

        \item étude qualitative : obtenir des informations sur (la monotonie de) une solution avec une condition initiale, sous la condition que les courbes des solutions ne se croisent pas.
    \end{itemize}

% =================================================
%\prsec{Calcul différentiel à plusieurs variables}{12}
% =================================================

%    \prsubsec {Fonctions de deux variables~:} domaine de définition, graphe.

   % \prsubsec {Dérivées partielles~:} fonctions partielles, dérivée partielles d'ordre 1 et 2, lemme de Schwarz.

%    \prsubsec {Points critiques~:} extrema locaux (maximum, minimum, point selle).

    %\prsubsec{Équations aux dérivées partielles~:}
    %\begin{itemize}
       % \item décrire des domaines faciles de $\R^{2}$ : rectangles, cercles, couronnes;

        %\item vérification qu'une fonction est solution d'une équation, mais pas besoin de savoir la résoudre.
    %\end{itemize}

% =================================================
\prsec{Primitives et intégrales}{13}
% =================================================
 \prsubsec{Notion de primitive :} comme "opération inverse" de la dérivée.

    \prsubsec{Intégrale de Riemann :} idée de la construction, interprétation comme surface sous une fonction positive, lien avec les primitives.

    \prsubsec{Règles de calcul~:} changement de variable, int\'egration par parties.

    \prsubsec{Applications~:} calculs des volumes, surfaces et longueur.

  %  \textit{La notion d'intégrale multiple n'est pas nécessaire pour traiter les exercices de calcul d'aire ou de volume.}

 \prsubsec{Intégration d'équations différentielles~:} au moyen de primitives, exemples du premier chapitre.

