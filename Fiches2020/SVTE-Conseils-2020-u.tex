% ----------------------------------------------------------------
%
%   Conseils généraux
%
% ----------------------------------------------------------------

% ~~~~~~~~~~~~~~~~~~~~~~~~~~~~~~~~~~~~~~~~~~~~~~~~~
\likesection{ Quelques conseils}
% ~~~~~~~~~~~~~~~~~~~~~~~~~~~~~~~~~~~~~~~~~~~~~~~~~

% =================================================
\subsection{Comment travailler}
% =================================================

\begin{enumerate}
\item  Il est important d'aller en cours et en TDs, car votre enseignant essaye oralement
    de mettre en évidence les aspects essentiels d'une manière qui n'est en général pas
    visible dans des notes prises par vos collègues.

\item N'hésitez pas à poser des questions. Vous avez peut-être peur qu'elles soient
    stupides, mais ce qui serait stupide serait de ne pas essayer de comprendre au mieux.
    Il ne faut pas hésiter à vous adresser à votre enseignant,
    que ce soit pendant ou après le cours ou le TD.

\item Il faut travailler toutes les semaines. Il n'est pas suffisant de le faire avant les examens,
    car il faut progressivement se fabriquer des automatismes. De plus, les notions nouvelles
    sont construites à partir des plus anciennes, qu'il faut avoir suffisamment comprises et
    manipulées.

\item C'est une très bonne idée de travailler en groupe, de discuter ainsi à plusieurs de
    vos difficultés.

\item Relisez rapidement un cours après y avoir assisté, afin de l'avoir encore assez
    fraîchement en tête. Lors de cette première relecture vous pouvez marquer en marge
    du texte les endroits qui vous posent problème, puis revenir dessus  plus attentivement
    lors d'une deuxième relecture.

\item Il est important de bien comprendre les définitions et les théorèmes,
    et d'être capable de les reproduire.
    Il ne s'agit pas de les apprendre par c\oe ur, mais de les mettre à l'épreuve lors de
    résolutions d'exercices. Ainsi, la première chose à faire lorsque l'on aborde un exercice est
    de comprendre toutes les notions en jeu.

\item Il ne faut pas avoir honte de faire des erreurs, au contraire il faut savoir en  profiter.
    Ainsi, lorsque vous découvrez que vous en avez faite une, il faut réfléchir à ce qui
    l'a causé : inattention, compréhension erronée d'une notion, confusion entre deux
    notions différentes.

\item Pour lutter convenablement contre les erreurs d'inattention, il faut toujours se relire.
    Pour lutter contre les confusions et les compréhensions erronées, il faut rédiger de la
    manière la plus claire possible, en bon français, en expliquant les théorèmes
    utilisés.

\item Afin d'indiquer que vous comprenez pourquoi un théorème peut être utilisé
    à un certain endroit, il faut être bien conscient de ses hypothèses. Il faut expliquer pourquoi
    celles-ci sont vérifiées dans votre contexte.

\item N'hésitez pas à lire d'autres présentations de la même théorie dans les ouvrages
    de la bibliothèque ou sur Internet. Vous découvrirez ainsi une diversité de points de vue,
    certains vous parlant plus que d'autres.
\end{enumerate}

% =================================================
\subsection{Quelques sites Internet qui peuvent vous intéresser}
% =================================================

\begin{itemize}
    \item    Le {\bf Portail des universités numériques thématiques}.
        Il s'agit d'une base de donnée
       de documents gratuits pour l'enseignement supérieure dans une multitude de matières.

         \url{http://www.universites-numeriques.fr}

    \item    Le site {\bf Exo7} où on trouve des feuilles d'exercices et des résolutions filmées d'exercices de mathématiques pour
        l'enseignement supérieur.

         \url{http://exo7.emath.fr}

    \item   Le site {\bf Images des Mathématiques} de la communauté mathématique française
        adressé au grand public. Permet de découvrir des sujets de recherche contemporains,
        des aspects de l'histoire des maths, des interactions des maths avec d'autres disciplines,
        des débats et bien d'autres choses.

        \url{http://images.math.cnrs.fr}

    \item Le site {\bf CultureMath}, ayant le même esprit que le précédent, mais en plus scolaire.

        \url{http://www.math.ens.fr/culturemath}

    \item {\bf Wikipedia}, encyclopédie générale en ligne. N'hésitez pas à aller y lire les
        explications données sur les notions du cours. Si vous lisez d'autres langues, c'est
        aussi un bon exercice de lire les explications dans celles-ci.

        \url{http://fr.wikipedia.org}
\end{itemize}


