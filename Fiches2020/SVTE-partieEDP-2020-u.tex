
% ~~~~~~~~~~~~~~~~~~~~~~~~~~~~~~~~~~~~~~~~~~~~~~~~~
\section{CALCUL DIFFÉRENTIEL À PLUSIEURS VARIABLES}
% ~~~~~~~~~~~~~~~~~~~~~~~~~~~~~~~~~~~~~~~~~~~~~~~~~


% =================================================
\subsection{Calculs d'extrema}
% =================================================

% -------------------------------------------------
\begin{exo}
    Calculer, quand elles existent, les dérivées partielles premières et secondes des fonctions suivantes~:
    \begin{examplescol}{2}
        \item $f(x,y) = x^{2}y+3xy-x^{3}+y^{2}$
        \item $g(x,y) = \ln(1+xy)$
        \item $h(x,y) = xe^{-(x^{2}+y^{2})}$
        \item $k(x,y) = \sqrt{x^{2}+y^{2}}$
    \end{examplescol}
\end{exo}

% -------------------------------------------------
\begin{exo}
    Déterminer les points stationnaires des fonctions suivantes et
    préciser pour chacun d'eux s'il s'agit d'un maximum local, d'un minimum
    local ou d'un col.
    \begin{examplescol}{3}
        \item $x^2y-\frac{x^2}{2}-y^2$
        \item $x^4+y^4-4(x-y)^2$
        \item $x^3 + x^2y + y^2$
        \item $2x^2 -y^2 +2xy-6x-6y+3$
        \item $3x^2 -y^2 +2xy-4x-4y+3$
        \item $4x^3 -3x+y^2 -4y-3$
        \item $3x^3 -x+3y^2 +3y-5$
        \item $\sqrt{1+x^2+y^2}$
        \item $x^3 +y^3 +3xy$
        \item $\exp(x^2 -y^2)$
        \item $2xy + (1/x^2)  + (1/y^2)$
        \item $2xy^2-x^2y+6x$
    \end{examplescol}
\end{exo}

% -------------------------------------------------
\begin{exo}
  On considère la fonction
    \begin{equation}
      w(x,y) = 1 + e^{2x}(x+y)^{2} \label{sol_2}
    \end{equation}
  \begin{enumerate}
    \item Montrer que l'ensemble des points critiques (ou stationnaires) de la fonction $w$ sont les points $(x,y)$ de $\mathbb{R}^{2}$ tels que $y=-x$. (Il y a donc une droite entière de points critiques.)

    {\em Dans la suite, on va essayer de déterminer la nature du point critique $(1,-1)$.}

    \item Calculer les dérivées partielles secondes de $w$, et évaluez-les en $(1,-1)$. Peut-on en déduire la nature du point critique $(1,-1)$~?

    \item Montrer que $w$ admet en fait un minimum en $(1,-1)$.\\
    ({\em Indication: Examiner le signe de $w(x,y) - w(1,-1)$ pour tout $(x,y) \in \mathbb{R}^{2}$.})
  \end{enumerate}
\end{exo}

% -------------------------------------------------
\begin{exo}

    \image{r}{5.5cm}{-14mm}{3mm}{drawings/distance_courbes.tikz}
    On se propose dans cet exercice de calculer la distance minimale entre les graphes de deux fonctions
    \begin{eqnarray*}
        f\left( s\right)  &=&-(s^{2}+3s+3) \\
        g\left( t\right)  &=&2-t
    \end{eqnarray*}

    \begin{enumerate}
        \item Vérifier que $(-1,1)$ est un  point stationnaire de la fonction à deux variables
        \begin{equation*}
            D\left( s,t\right) =\left( t-s\right) ^{2}+\left( 2-t+s^{2}+3s+3\right) ^{2}
        \end{equation*}
        et déterminer s'il s'agit d'un maximum local, minimum local ou col.
        ({\em Indication~: On se rappellera que si $u$ est une fonction à deux variables $s$ et $t$, on a
        $\partial_s u^2 = 2\,u\, \partial_s u$.})
    \end{enumerate}

    \begin{enumerate}\setcounter{enumi}{1}
        \item Vérifier (sans faire référence au dessin) que le point $\left( -1,-1\right) $ fait partie du graphe $\Gamma _{f}$ de $f$.
        Puis vérifier que $\left(1,1\right)$ fait partie du graphe $\Gamma _{g}$ de $g$.
        \item On remarquera que $D(s,t)$ représente le carré de la distance entre le point $\left( s,f\left( s\right) \right) $ de $\Gamma _{f}$  et le point $\left( t,g\left( t\right) \right) $ de $\Gamma _{g}$.
        En admettant que le minimum local de $D$ trouvé auparavant est global, calculer la distance minimale entre les deux graphes.
    \end{enumerate}
\end{exo}

% =================================================
\subsection{Extrema et équations différentielles}
% =================================================


% -------------------------------------------------
\begin{exo}
    Dans un milieu naturel, à tout instant $t \geq 0$, on désigne par $X(t)$ la quantité de poissons d'un lac et par $Y(t)$ le nombre d'habitants vivant aux abords du lac et se nourrissant principalement de poissons.
    \begin{enumerate}
        \item On représente cette interaction en utilisant le modèle de Lotka-Volterra:
        \begin{eqnarray}
            & & \left\{ \begin{array}{ll} X'(t) = aX(t) - bX(t)Y(t) \\
            Y'(t) = -cY(t) + dX(t)Y(t)
        \end{array} \right.  \label{Volterra}\\
        & & \left\{ \begin{array}{ll} X(0) = x_0 \\
        Y(0)=y_0
    \end{array} \right.  \label{ciVolterra}
    \end{eqnarray}
    où $a,b,c,d>0$, et où $x_0>0$ et $y_0>0$ désignent les effectifs de population au début de l'observation.
    \begin{enumerate}
        \item Expliquer comment évolue la quantité de poissons en l'absence d'habitants; que représente $a$? Que devient le taux d'accroissement de la population de  poissons en présence d'habitants?
        \item Expliquer comment évolue le nombre d'habitants en l'absence de poissons; que représente $c$? Que devient le taux d'accroissement des habitants en présence de poissons?
    \end{enumerate}
    \item\label{Volterra-constant} Déterminer les solutions constantes du système différentiel (\ref{Volterra}).
    \item Donner le domaine de définition $\mathcal{D}$ de la fonction
        $$
            L(x,y) = -a\ln(y) + by - c\ln(x) + dx.
        $$
    \item Quels sont les points critiques de la fonction $L(x,y)$ sur $\mathcal{D}$? Déterminer leur nature. Qu'observe-t-on par rapport à la question (\ref{Volterra-constant})?
    \item Soit $X(t)$, $Y(t)$ des fonctions satisfaisant (\ref{Volterra}-\ref{ciVolterra}). Calculer $L(X(t),Y(t))$ en fonction de $x_0$ et $y_0$ ({\it indication: dériver la fonction}).
    \item On fixe $a=b=c=d=1$. Soit $K \in \R $. On considère la courbe $\mathcal{C}_K$ d'équation $L(x,y)=K$.
        \begin{enumerate}
            \item Montrer que la fonction $L$ admet un minimum global strict en $(1,1)$ ({\it indication: on pourra étudier la fonction $x\mapsto x-\ln(x)$}).
            \item Que peut-on dire de l'ensemble des couples $(x,y)$ tels que $L(x,y)=K$ lorsque $K=2$ et lorsque $K < 2$?
            \item L'allure des courbes $\mathcal{C}_K$, pour $K>2$, est la suivante:
                \begin{center}
                    \begin{tikzpicture}[x=0.42cm, y=0.42cm, xmin=-2,xmax=16,ymin=-2,ymax=16]
    \grille \fenetre \axes

    % les valeurs aux axes
    \foreach \x in {2,4,...,12}
    {
        \draw (\x,0)--(\x,-0.1) (\x,0) node[below, opacity=0.91]{$\x$};
        \draw (0,\x)--(-0.1,\x) (0,\x) node[left, opacity=0.91]{$\x$};
    }

    % ------------------------------------------------------
    % les courbes pour K=1,2,...
    % a=1 b=1 c=1 d=1
    \filldraw[red] (1,1) circle(1pt);

    \path (3.8077 , 3.8077) node[red, rotate=-45] {$<$} node[above right, scale=0.7] {K=3};
    \draw[red] plot coordinates { (3.807730,3.807730) (3.806977,3.808483) (3.805471,3.809988) (3.802459,3.812998) (3.796433,3.819015) (3.784378,3.831030) (3.123526,4.449656) (2.520235,4.943605) (1.976490,5.317166) (1.460069,5.578777) (1.158300,5.664214) (0.915242,5.673367) (0.736021,5.626333) (0.595870,5.539737) (0.481395,5.418807) (0.387633,5.267647) (0.310774,5.089013) (0.247862,4.884772) (0.196525,4.656075) (0.154814,4.403430) (0.121109,4.126693) (0.094048,3.824942) (0.072481,3.496139) (0.055423,3.136228) (0.042240,2.744284) (0.032549,2.329445) (0.024908,1.826082) (0.021638,1.475208) (0.020215,1.212454) (0.019810,1.000282) (0.020158,0.826506) (0.021149,0.683451) (0.022829,0.561259) (0.025306,0.457600) (0.028764,0.370160) (0.033488,0.296879) (0.039906,0.235905) (0.048664,0.185571) (0.060753,0.144381) (0.077743,0.110993) (0.102232,0.084212) (0.138826,0.062974) (0.193115,0.047040) (0.261659,0.036802) (0.345919,0.030082) (0.448564,0.025591) (0.572587,0.022628) (0.721229,0.020805) (0.898041,0.019923) (1.106957,0.019918) (1.352439,0.020859) (1.646307,0.023039) (2.007319,0.027225) (2.458431,0.035180) (3.060663,0.052506) (3.530293,0.074420) (3.942348,0.103604) (4.387354,0.152565) (4.760257,0.217960) (5.059152,0.300244) (5.294936,0.402088) (5.473874,0.527094) (5.597799,0.679428) (5.665566,0.863898) (5.673684,1.086173) (5.616334,1.353105) (5.481980,1.679105) (5.245402,2.090906) (4.869391,2.616943) (4.276703,3.316854) (3.529097,4.079204) (2.869358,4.666495) (2.293607,5.108849) (1.766889,5.436897) (1.374297,5.609456) (1.057428,5.676018) (0.840904,5.660728) (0.679351,5.597754) (0.549639,5.497522) (0.443555,5.364711) (0.356610,5.202839) (0.285364,5.014295) (0.227104,4.800660) (0.179635,4.562837) (0.141141,4.301091) (0.110109,4.015006) (0.085262,3.703293) (0.065516,3.363311) (0.049942,2.989683) (0.038220,2.590650) (0.029536,2.159600) (0.024218,1.764135) (0.021364,1.435313) (0.020108,1.180624) (0.019817,0.974302) (0.020258,0.805151) (0.021341,0.665182) (0.023125,0.545727) (0.025726,0.444464) (0.029341,0.359120) (0.034271,0.287665) (0.040971,0.228273) (0.050125,0.179302) (0.062787,0.139279) (0.080637,0.106882) (0.106475,0.080936) (0.145329,0.060394) (0.201393,0.045409) (0.271863,0.035747) (0.358391,0.029380) (0.463690,0.025122) (0.590779,0.022328) (0.742940,0.020638) (0.923767,0.019873) (1.137254,0.019986) (1.387952,0.021065) (1.689762,0.023454) (2.061047,0.028001) (2.526905,0.036709) (3.162674,0.056496) (3.619070,0.079761) (4.020156,0.110593) (4.449234,0.161485) (4.809130,0.229094) (5.097865,0.314034) (5.324934,0.419063) (5.495684,0.547842) (5.611417,0.704618) (5.670617,0.894311) (5.669457,1.122739) (5.601745,1.396969) (5.453972,1.734318) (5.199490,2.160950) (4.798038,2.707541) (4.156838,3.446651) (3.418559,4.183025) (2.776466,4.742636) (2.209769,5.166679) (1.688013,5.477522) (1.323506,5.625400) (1.026769,5.677487) (0.818839,5.655199) (0.662378,5.587580) (0.535765,5.483193) (0.432190,5.346738) (0.347293,5.181558) (0.277736,4.989934) (0.220879,4.773362) (0.174575,4.532667) (0.155748,4.409923) };

    \path (5.2305 , 5.2305) node[red, rotate=-45] {$<$} node[above right, scale=0.7] {K=4};
    \draw[red] plot coordinates { (5.230539,5.230539) (5.230537,5.230541) (5.230532,5.230545) (5.230523,5.230554) (5.230505,5.230572) (5.230470,5.230608) (5.230399,5.230679) (5.230257,5.230821) (5.229972,5.231105) (5.229403,5.231674) (5.228266,5.232812) (5.225990,5.235086) (5.221440,5.239634) (5.212339,5.248724) (4.553408,5.887317) (3.645146,6.702829) (2.827129,7.360279) (2.027947,7.897697) (1.619586,8.107412) (1.284400,8.225135) (1.033863,8.263314) (0.839950,8.247610) (0.686485,8.192628) (0.559522,8.104242) (0.453890,7.985942) (0.366182,7.840551) (0.293576,7.670196) (0.233722,7.476463) (0.184633,7.260479) (0.144619,7.022967) (0.112230,6.764274) (0.086221,6.484386) (0.065520,6.182926) (0.049207,5.859135) (0.036494,5.511838) (0.026705,5.139379) (0.019270,4.739505) (0.013706,4.309167) (0.009610,3.844069) (0.006650,3.337515) (0.004552,2.776290) (0.003323,2.255003) (0.002711,1.858491) (0.002375,1.536397) (0.002202,1.271298) (0.002137,1.052325) (0.002153,0.871216) (0.002239,0.721341) (0.002394,0.594136) (0.002628,0.485778) (0.002956,0.394083) (0.003404,0.317004) (0.004007,0.252678) (0.004818,0.199416) (0.005913,0.155692) (0.007407,0.120130) (0.009470,0.091503) (0.012361,0.068717) (0.016491,0.050805) (0.022516,0.036919) (0.031533,0.026320) (0.045450,0.018370) (0.067770,0.012524) (0.098278,0.008871) (0.138138,0.006553) (0.189136,0.005029) (0.253354,0.003999) (0.333114,0.003292) (0.430975,0.002804) (0.549745,0.002475) (0.692495,0.002266) (0.862573,0.002156) (1.063618,0.002138) (1.299579,0.002216) (1.578511,0.002411) (1.917381,0.002787) (2.329684,0.003467) (2.832730,0.004721) (3.451623,0.007212) (4.237013,0.012960) (4.835774,0.020828) (5.357142,0.032024) (5.925448,0.052148) (6.420582,0.081297) (6.835830,0.120260) (7.185492,0.170692) (7.480516,0.234806) (7.726883,0.315158) (7.927317,0.414620) (8.082448,0.536414) (8.191388,0.684185) (8.251981,0.862070) (8.260881,1.074816) (8.213487,1.327925) (8.101629,1.632668) (7.907499,2.010922) (7.608000,2.482471) (7.170616,3.074551) (6.544317,3.828865) (5.607917,4.846545) (4.456739,5.977735) (3.554326,6.779802) (2.743559,7.422185) (2.114881,7.846156) (1.602332,8.114837) (1.266764,8.229434) (1.020277,8.263683) (0.829281,8.245165) (0.677972,8.188073) (0.552425,8.097738) (0.447992,7.977693) (0.361291,7.830707) (0.289536,7.658869) (0.230400,7.463734) (0.181918,7.246405) (0.142414,7.007579) (0.110453,6.747582) (0.084800,6.466378) (0.064396,6.163566) (0.048326,5.838364) (0.035811,5.489567) (0.026184,5.115488) (0.018877,4.713833) (0.013415,4.281491) (0.009398,3.814063) (0.006498,3.304629) (0.004445,2.739221) (0.003275,2.228869) (0.002685,1.837568) (0.002361,1.519230) (0.002195,1.257132) (0.002136,1.040613) (0.002157,0.861526) (0.002246,0.713320) (0.002405,0.587277) (0.002644,0.479957) (0.002979,0.389174) (0.003435,0.312893) (0.004049,0.249262) (0.004874,0.196601) (0.005989,0.153392) (0.007511,0.118270) (0.009614,0.090015) (0.012565,0.067540) (0.016784,0.049887) (0.022949,0.036213) (0.032191,0.025786) (0.046484,0.017973) (0.069322,0.012260) (0.100349,0.008705) (0.140813,0.006445) (0.150473,0.006088) };

    \path (6.7819 , 6.7819) node[red, rotate=-45] {$<$} node[above right, scale=0.7] {K=5};
    \draw[red] plot coordinates { (6.781852,6.781852) (6.781852,6.781852) (6.781852,6.781853) (6.781852,6.781853) (6.781851,6.781853) (6.781850,6.781854) (6.781848,6.781857) (6.781844,6.781861) (6.781835,6.781869) (6.781818,6.781887) (6.781784,6.781921) (6.781715,6.781990) (6.781577,6.782127) (6.781302,6.782402) (6.780752,6.782952) (6.779653,6.784052) (6.777453,6.786251) (6.773054,6.790649) (6.764256,6.799441) (6.746664,6.817009) (5.415912,8.100567) (4.244492,9.150098) (3.177113,10.018496) (2.513463,10.494227) (1.959666,10.830696) (1.562976,11.018384) (1.262829,11.113928) (1.027936,11.145892) (0.840890,11.130729) (0.690342,11.079379) (0.564409,10.996385) (0.458859,10.884690) (0.370708,10.746760) (0.297401,10.584522) (0.236752,10.399464) (0.186874,10.192680) (0.146135,9.964913) (0.113114,9.716566) (0.086581,9.447720) (0.065464,9.158130) (0.048838,8.847215) (0.035901,8.514039) (0.025969,8.157287) (0.018454,7.775216) (0.012861,7.365606) (0.008774,6.925685) (0.005848,6.452030) (0.003800,5.940442) (0.002404,5.385760) (0.001478,4.781572) (0.000885,4.119634) (0.000517,3.388112) (0.000346,2.792477) (0.000258,2.309872) (0.000209,1.912131) (0.000182,1.583225) (0.000167,1.310993) (0.000161,1.085605) (0.000162,0.898978) (0.000167,0.744439) (0.000178,0.613971) (0.000195,0.502661) (0.000218,0.408354) (0.000250,0.328978) (0.000293,0.262646) (0.000351,0.207644) (0.000429,0.162420) (0.000535,0.125576) (0.000680,0.095862) (0.000882,0.072161) (0.001168,0.053486) (0.001581,0.038969) (0.002189,0.027855) (0.003106,0.019488) (0.004526,0.013308) (0.006797,0.008842) (0.010562,0.005694) (0.017076,0.003537) (0.028064,0.002173) (0.044061,0.001405) (0.066138,0.000957) (0.095693,0.000681) (0.134355,0.000504) (0.183971,0.000387) (0.246602,0.000307) (0.324539,0.000252) (0.420310,0.000214) (0.536691,0.000189) (0.676719,0.000172) (0.843699,0.000163) (1.041221,0.000161) (1.273163,0.000166) (1.546125,0.000180) (1.877423,0.000206) (2.279752,0.000254) (2.768432,0.000341) (3.362292,0.000508) (4.085092,0.000862) (4.969639,0.001717) (6.077538,0.004262) (6.804908,0.007906) (7.417935,0.013463) (8.075575,0.024125) (8.649724,0.040660) (9.133684,0.063947) (9.545576,0.095317) (9.899949,0.136498) (10.205409,0.189498) (10.466658,0.256577) (10.685997,0.340260) (10.864111,0.443370) (11.000476,0.569059) (11.093563,0.720861) (11.140934,0.902746) (11.139258,1.119193) (11.084267,1.375297) (10.967272,1.684451) (10.772193,2.065530) (10.478954,2.536612) (10.061016,3.121255) (9.482009,3.851119) (8.687767,4.772357) (7.572186,5.975298) (6.298647,7.259345) (5.010289,8.473518) (3.878877,9.459002) (2.801414,10.295815) (2.236412,10.671394) (1.768148,10.928528) (1.420010,11.070286) (1.151610,11.135013) (0.939677,11.144409) (0.770015,11.112036) (0.631424,11.046257) (0.514974,10.950104) (0.417532,10.826432) (0.336297,10.677444) (0.268890,10.504863) (0.213265,10.309996) (0.167653,10.093777) (0.130522,9.856795) (0.100539,9.599310) (0.076546,9.321257) (0.057540,9.022243) (0.042652,8.701532) (0.031135,8.358022) (0.022348,7.990213) (0.015747,7.596158) (0.010873,7.173401) (0.007343,6.718892) (0.004840,6.228881) (0.003108,5.698765) (0.001942,5.122873) (0.001179,4.494081) (0.000698,3.802910) (0.000428,3.114180) (0.000301,2.572579) (0.000234,2.128944) (0.000196,1.762578) (0.000174,1.459458) (0.000164,1.208528) (0.000161,1.000762) (0.000164,0.828724) (0.000171,0.686264) (0.000184,0.564206) (0.000204,0.460426) (0.000230,0.372738) (0.000267,0.299154) (0.000316,0.237861) (0.000381,0.187216) (0.000470,0.145734) (0.000527,0.127641) };

    \path (8.4273 , 8.4273) node[red, rotate=-45] {$<$} node[above right, scale=0.7] {K=6};
    \draw[red] plot coordinates { (8.427324,8.427324) (8.427324,8.427324) (8.427324,8.427324) (8.427324,8.427324) (8.427324,8.427324) (8.427324,8.427324) (8.427324,8.427324) (8.427324,8.427325) (8.427323,8.427325) (8.427322,8.427326) (8.427319,8.427329) (8.427315,8.427334) (8.427305,8.427343) (8.427286,8.427362) (8.427248,8.427400) (8.427173,8.427476) (8.427021,8.427627) (8.426718,8.427930) (8.426112,8.428536) (8.424900,8.429748) (8.422477,8.432171) (8.417630,8.437017) (8.407938,8.446705) (8.388557,8.466067) (8.349816,8.504736) (6.675788,10.129878) (5.179487,11.499184) (3.809152,12.658252) (3.029606,13.254881) (2.378492,13.696828) (1.902402,13.969264) (1.538857,14.132338) (1.253219,14.218747) (1.025272,14.247991) (0.841568,14.233211) (0.692508,14.183850) (0.567115,14.103827) (0.461601,13.995790) (0.373199,13.861999) (0.299500,13.704256) (0.238408,13.523975) (0.188093,13.322211) (0.146954,13.099690) (0.113591,12.856821) (0.086777,12.593695) (0.065443,12.310090) (0.048658,12.005451) (0.035615,11.678873) (0.025620,11.329069) (0.018079,10.954322) (0.012486,10.552434) (0.008419,10.120642) (0.005527,9.655520) (0.003520,9.152848) (0.002166,8.607444) (0.001283,8.012953) (0.000728,7.361586) (0.000393,6.643809) (0.000202,5.848018) (0.000098,4.960230) (0.000049,4.083603) (0.000030,3.378086) (0.000020,2.796770) (0.000015,2.315877) (0.000012,1.917753) (0.000010,1.588090) (0.000010,1.315103) (0.000009,1.089043) (0.000009,0.901842) (0.000010,0.746821) (0.000010,0.616016) (0.000011,0.504402) (0.000013,0.409826) (0.000014,0.330214) (0.000017,0.263675) (0.000020,0.208494) (0.000025,0.163115) (0.000031,0.126140) (0.000039,0.096314) (0.000050,0.072518) (0.000067,0.053765) (0.000090,0.039184) (0.000125,0.028016) (0.000177,0.019606) (0.000257,0.013392) (0.000386,0.008899) (0.000597,0.005730) (0.000960,0.003557) (0.001613,0.002116) (0.002852,0.001197) (0.005362,0.000638) (0.010296,0.000334) (0.018312,0.000189) (0.030235,0.000116) (0.047121,0.000076) (0.070245,0.000052) (0.101077,0.000037) (0.141288,0.000028) (0.192757,0.000021) (0.257581,0.000017) (0.338084,0.000014) (0.436832,0.000012) (0.556641,0.000011) (0.700587,0.000010) (0.872019,0.000009) (1.074566,0.000009) (1.312151,0.000010) (1.593414,0.000010) (1.934799,0.000012) (2.349324,0.000015) (2.852670,0.000021) (3.463887,0.000031) (4.206173,0.000054) (5.107961,0.000110) (6.205091,0.000271) (7.548478,0.000855) (8.624288,0.002196) (9.428214,0.004498) (10.096159,0.008223) (10.798722,0.015636) (11.409631,0.027588) (11.924803,0.044959) (12.364653,0.068948) (12.745404,0.101080) (13.076846,0.143126) (13.364577,0.197089) (13.611637,0.265208) (13.819382,0.349985) (13.987942,0.454203) (14.116471,0.580962) (14.203284,0.733713) (14.245920,0.916301) (14.241164,1.133025) (14.185011,1.388715) (14.068925,1.697163) (13.877971,2.075876) (13.593831,2.541830) (13.192963,3.116637) (12.644548,3.828246) (11.906964,4.713819) (10.919889,5.826227) (9.574497,7.260171) (7.907854,8.944165) (6.223774,10.554296) (4.775179,11.854085) (3.681458,12.761439) (2.775514,13.436454) (2.187041,13.814379) (1.757660,14.041501) (1.425821,14.173459) (1.163342,14.236957) (0.953017,14.248686) (0.783041,14.220214) (0.643638,14.159274) (0.525930,14.068588) (0.427048,13.950791) (0.344348,13.807955) (0.275540,13.641733) (0.218634,13.453403) (0.171888,13.243897) (0.133778,13.013822) (0.102971,12.763468) (0.078300,12.492814) (0.058750,12.201515) (0.043437,11.888889) (0.031597,11.553895) (0.022574,11.195093) (0.015807,10.810595) (0.010824,10.398005) (0.007229,9.954327) (0.004694,9.475859) (0.002954,8.958046) (0.001793,8.395300) (0.001046,7.780769) (0.000583,7.106056) (0.000309,6.360899) (0.000155,5.532844) (0.000074,4.606955) (0.000043,3.889453) };

    % le poisson
    \node[scale=0.7] at (14.5,-0.7) {
        \begin{tikzpicture}
            \filldraw[fill=red!20, draw=black, yscale=-1] (0,0) svg "M 0.04,7.85 C -0.26,7.41 -0.17,7.35 -0.26,7.16 C -0.30,7.09 -0.37,6.83 -0.42,6.58 C -0.51,6.07 -0.50,6.04 -0.02,5.86 C 0.25,5.75 0.27,5.74 0.13,5.71 C -1.55,5.33 -2.03,5.21 -2.72,4.94 C -3.45,4.66 -3.08,4.48 -4.13,5.55 C -4.58,6.00 -5.02,6.41 -5.12,6.46 C -5.52,6.70 -5.97,6.29 -6.25,6.17 C -6.37,6.17 -7.53,5.01 -7.66,4.77 C -7.79,4.54 -7.98,4.49 -8.00,4.37 C -8.03,4.26 -8.35,3.96 -8.38,3.85 C -8.41,3.75 -8.73,3.34 -8.76,3.24 C -8.79,3.12 -9.11,2.82 -9.14,2.70 C -9.16,2.61 -9.50,2.14 -9.48,2.04 C -9.42,1.91 -9.07,1.79 -8.14,1.59 C -8.00,1.56 -8.00,1.55 -8.27,1.27 C -8.99,0.49 -9.25,0.25 -9.42,0.21 C -9.52,0.18 -10.15,0.21 -10.81,0.27 C -12.34,0.40 -13.85,0.41 -14.29,0.28 C -14.78,0.13 -15.00,-0.23 -14.81,-0.58 C -14.77,-0.65 -14.58,-0.78 -14.38,-0.87 C -13.26,-1.40 -12.01,-2.47 -11.64,-3.22 C -11.50,-3.50 -11.50,-3.55 -11.50,-5.65 C -11.50,-8.10 -11.49,-8.16 -11.08,-8.19 C -10.88,-8.21 -10.80,-8.17 -10.55,-7.94 C -9.86,-7.31 -9.47,-6.48 -9.00,-4.74 C -8.73,-3.70 -8.49,-3.15 -8.12,-2.76 C -7.61,-2.19 -6.53,-1.72 -5.94,-1.80 C -5.63,-1.84 -5.17,-2.07 -5.17,-2.18 C -5.17,-2.27 -5.77,-2.65 -6.05,-2.94 C -6.35,-3.25 -6.23,-3.13 -5.80,-4.04 C -5.51,-4.65 -5.27,-4.81 -5.23,-4.95 C -5.23,-5.02 -5.11,-5.15 -4.68,-5.58 C -4.40,-5.85 -4.53,-5.89 -4.30,-6.02 C -4.07,-6.15 -3.98,-6.40 -3.90,-6.44 C -3.64,-6.55 -3.12,-7.11 -2.43,-7.71 C -2.34,-7.79 -2.15,-7.95 -2.06,-8.03 C -1.94,-8.02 -1.71,-8.00 -1.59,-7.99 C -0.38,-7.88 0.92,-7.12 2.16,-5.81 C 2.31,-5.65 2.61,-5.33 2.76,-5.18 C 2.89,-5.18 3.15,-5.18 3.29,-5.18 C 6.13,-5.17 9.09,-4.36 11.32,-2.53 C 12.16,-1.84 13.19,-0.66 13.73,0.23 C 13.95,0.59 13.97,0.67 13.86,0.90 C 13.76,1.12 14.02,1.33 14.08,1.52 C 14.16,1.74 13.58,2.39 12.71,3.05 C 11.72,3.79 10.35,4.61 9.98,4.69 C 9.70,4.75 9.90,4.79 9.38,4.97 C 8.73,5.21 8.68,5.28 7.10,5.54 C 6.71,5.60 5.08,5.89 4.45,5.98 C 3.89,6.06 3.39,6.14 2.88,6.20 C 2.54,6.25 2.52,6.28 2.39,6.93 C 2.26,7.61 2.06,7.89 1.93,7.40 C 1.88,7.20 1.85,7.15 1.41,7.57 C 0.74,8.22 0.36,8.44 0.04,7.85 Z";
            \begin{scope}
                \clip (0.9,0) circle (0.1);
                \filldraw[fill=white, draw=black] (0.9,0) circle (0.1);
                \filldraw[fill=black] (1,0) circle (0.1);
            \end{scope}
        \end{tikzpicture}
    };

    % le pêcheur
    \node[scale=2] at (-0.9, 13.5) {
        \begin{tikzpicture}
            %\fill[fill=black, draw=black, yscale=-1] (0,0) svg "M -0.31,-5.14 C -0.25,-5.14 -0.19,-5.14 -0.13,-5.14 C 0.27,-5.09 0.57,-4.94 0.66,-4.58 C 0.61,-4.31 0.60,-4.04 0.48,-3.79 C 0.51,-3.74 0.57,-3.74 0.61,-3.71 C 0.59,-3.62 0.64,-3.59 0.64,-3.51 C 1.14,-2.93 1.32,-1.44 0.94,-0.65 C 1.00,-0.44 1.04,-0.27 0.97,-0.09 C 0.97,0.04 1.06,0.07 1.07,0.19 C 1.06,0.33 0.95,0.36 0.89,0.45 C 0.92,0.95 0.64,1.29 0.69,1.75 C 0.72,2.06 0.82,2.31 0.84,2.72 C 0.87,3.27 0.99,3.80 1.02,4.28 C 1.04,4.63 1.12,4.89 0.97,5.12 C 0.37,5.12 -0.18,5.14 -0.61,4.94 C -0.60,4.64 -0.95,4.82 -1.00,4.63 C -0.96,4.39 -0.67,4.52 -0.51,4.46 C -0.33,4.39 -0.32,4.21 -0.18,4.12 C -0.19,3.67 -0.23,3.35 -0.16,2.92 C -0.68,2.36 -0.62,1.11 -0.92,0.40 C -1.01,0.44 -1.15,0.42 -1.28,0.42 C -1.59,0.21 -1.33,-0.71 -1.28,-1.16 C -1.47,-1.29 -1.81,-1.28 -1.84,-1.57 C -1.84,-1.63 -1.90,-1.54 -1.97,-1.57 C -1.96,-1.72 -2.14,-1.87 -2.25,-1.98 C -2.74,-2.46 -3.41,-2.94 -3.98,-3.43 C -4.62,-3.98 -5.29,-4.50 -5.87,-5.09 C -5.30,-4.75 -5.35,-4.80 -5.13,-4.55 C -4.29,-3.67 -2.99,-2.74 -1.92,-1.85 C -1.92,-1.96 -1.90,-2.07 -1.84,-2.18 C -1.72,-2.22 -1.68,-2.22 -1.53,-2.21 C -1.39,-2.07 -1.36,-1.81 -1.18,-1.72 C -1.08,-2.09 -0.96,-2.46 -0.74,-2.82 C -0.56,-3.12 -0.26,-3.39 -0.59,-3.66 C -0.53,-3.84 -0.63,-4.03 -0.74,-4.12 C -0.75,-4.27 -0.60,-4.28 -0.67,-4.43 C -0.71,-4.48 -0.84,-4.44 -0.87,-4.50 C -0.84,-4.61 -0.68,-4.58 -0.61,-4.66 C -0.60,-4.90 -0.49,-5.06 -0.31,-5.14 Z";
            \path[fill=blue!35, draw=blue, ultra thin, yscale=-1] (0,0) svg "M -0.31,-20.26 C -0.25,-20.26 -0.19,-20.26 -0.13,-20.26 C 0.27,-20.21 0.57,-20.06 0.66,-19.70 C 0.61,-19.43 0.60,-19.16 0.48,-18.91 C 0.51,-18.86 0.57,-18.86 0.61,-18.83 C 0.59,-18.74 0.64,-18.71 0.64,-18.63 C 1.14,-18.05 1.32,-16.56 0.94,-15.77 C 1.00,-15.56 1.04,-15.39 0.97,-15.21 C 0.97,-15.08 1.06,-15.05 1.07,-14.93 C 1.06,-14.79 0.95,-14.76 0.89,-14.67 C 0.92,-14.17 0.64,-13.83 0.69,-13.37 C 0.72,-13.06 0.82,-12.81 0.84,-12.40 C 0.87,-11.85 0.99,-11.32 1.02,-10.84 C 1.04,-10.49 1.12,-10.23 0.97,-10.00 C 0.37,-10.00 -0.18,-9.98 -0.61,-10.18 C -0.60,-10.48 -0.95,-10.30 -1.00,-10.49 C -0.96,-10.73 -0.67,-10.60 -0.51,-10.66 C -0.33,-10.73 -0.32,-10.91 -0.18,-11.00 C -0.19,-11.45 -0.23,-11.77 -0.16,-12.20 C -0.68,-12.76 -0.62,-14.01 -0.92,-14.72 C -1.01,-14.68 -1.15,-14.70 -1.28,-14.70 C -1.59,-14.91 -1.33,-15.83 -1.28,-16.28 C -1.47,-16.41 -1.81,-16.40 -1.84,-16.69 C -1.84,-16.75 -1.90,-16.66 -1.97,-16.69 C -1.96,-16.84 -2.14,-16.99 -2.25,-17.10 C -2.74,-17.58 -3.41,-18.06 -3.98,-18.55 C -4.34,-18.86 -4.70,-19.15 -5.06,-19.46 C -5.07,-19.47 -5.14,-14.24 -5.15,-14.25 C -5.15,-14.25 -5.23,-14.25 -5.23,-14.28 C -5.23,-14.28 -5.15,-14.26 -5.16,-14.27 C -5.16,-14.83 -5.11,-19.51 -5.13,-19.52 C -5.38,-19.74 -5.63,-19.97 -5.87,-20.21 C -5.30,-19.87 -5.35,-19.92 -5.13,-19.67 C -4.29,-18.79 -2.99,-17.86 -1.92,-16.97 C -1.92,-17.08 -1.90,-17.19 -1.84,-17.30 C -1.72,-17.34 -1.68,-17.34 -1.53,-17.33 C -1.39,-17.19 -1.36,-16.93 -1.18,-16.84 C -1.08,-17.21 -0.96,-17.58 -0.74,-17.94 C -0.56,-18.24 -0.26,-18.51 -0.59,-18.78 C -0.53,-18.96 -0.63,-19.15 -0.74,-19.24 C -0.75,-19.39 -0.60,-19.40 -0.67,-19.55 C -0.71,-19.60 -0.84,-19.56 -0.87,-19.62 C -0.84,-19.73 -0.68,-19.70 -0.61,-19.78 C -0.60,-20.02 -0.49,-20.18 -0.31,-20.26 Z";
        \end{tikzpicture}
    };

\end{tikzpicture}




                \end{center}
            Se peut-il que le nombre d'habitants devienne très grand pour un temps assez long ?
        \end{enumerate}
    \end{enumerate}
\end{exo}

% -------------------------------------------------
\begin{exo}
    Soient $a<0$ et $b \in \R $. On considère le système différentiel
    \begin{eqnarray}
        & & \left\{ \begin{array}{ll} x'(t) = ax(t) - by(t) \\
        y'(t) = bx(t) + ay(t)
    \end{array} \right.  \label{S}\\
    & & \left\{ \begin{array}{ll} x(0) = x_0 \\
    y(0)=y_0
\end{array} \right.  \label{ciS}
\end{eqnarray}
On suppose qu'il existe des fonctions $x,y:\R _{+} \rightarrow \R $ qui satisfont le système (\ref{S})-(\ref{ciS}).
\begin{enumerate}
    \item Montrer que la fonction $H(t) = \frac{1}{2}(x^{2}(t)+y^{2}(t))$ satisfait l'équation différentielle
    $$
        H'(t)=2aH(t).
    $$
    \item Montrer qu'il existe un réel $K$ que l'on exprimera en fonction de $x_{0},y_{0}$ tel que $H(t)=Ke^{2at}$ puis que
    $$
        \lim_{t \rightarrow +\infty}H(t) = 0.
    $$
    \item En déduire que  $\lim_{t \rightarrow +\infty} x(t) = \lim_{t \rightarrow +\infty} y(t) = 0$.
    \item Montrer que $(0,0)$ est un minimum global pour la fonction $H(x,y) = \frac{1}{2}(x^2+y^2)$, puis vérifier que $(0,0)$ satisfait aux conditions d'un minimum local.
\end{enumerate}
\end{exo}


% =================================================
\subsection{Extrema et équations aux dérivées partielles}
% =================================================


% -------------------------------------------------
% New : Question 3 du exam-mar12.pdf (Equation de Burgers, exo de Jean-Paul..)
% -------------------------------------------------
\begin{exo}
    On considère la fonction $w$ à deux variables définie par~:
    $$
        w(x,t) = xt + \frac{x^3}{6} +1.
    $$
    \begin{enumerate}
        \item Déterminer les points critiques (ou stationnaires) de $w$ et préciser leur nature.
        \item Montrer que $w$ est solution de l'équation~:
        $$
            \partial_t w = \partial^2_x w
        $$
    \end{enumerate}

    \noindent On considère à présent la fonction $u$ à deux variables définie par~:
    \begin{equation}
        u = \frac{2\, \partial_x w}{w} \label{defu}
    \end{equation}

    \begin{enumerate}\setcounter{enumi}{2}
        \item Montrer que $u$ est bien définie pour tous réels $x$ et $t$ positifs.
        \item Montrer que pour tous réels $x$ et $t$ positifs, on a~:
        $$
            u(x,t) = \frac{12t+6x^2}{6xt+x^3+6}.
        $$
        \item Prouver que $u$ est solution de l'équation~:
        $$
            \partial_t u - u\,\partial_x u = \partial^2_x u\,.
        $$
        \textit{Indication~: On peut montrer (e) soit à partir de (d), soit à partir de (\ref{defu}) et (b).}
    \end{enumerate}
    Cette équation est un cas particulier de l'équation de Burgers, équation issue de la mécanique des fluides.
\end{exo}

% -------------------------------------------------
\begin{exo}
    On considère l'équation aux dérivées partielles
    \begin{equation}
        \partial_{x}^{2}u(x,y) + \partial_{y}^{2}u(x,y) + \partial_{x}u(x,y) + \partial_{y}u(x,y) = F(x,y)
        \label{elliptic}
    \end{equation}
    où $F$ est une fonction définie sur le domaine
    $$
        \Omega_{R} = \{(x,y) \in \R ^{2} \ \ : \ \ x^2 + y^2 < R^{2} \},
    $$
    avec $R>0$. On suppose que la fonction $F$ vérifie $F(x,y)>0$ pour tout $(x,y) \in \Omega_{R}$.\\
    On admet qu'il existe une fonction $(x,y) \mapsto u(x,y)$ qui est continue sur le domaine
    $$
        D_{R} = \{(x,y) \in \R ^{2} \ \ : \ \ x^2 + y^2 \leq R^{2} \}
    $$
    et qui vérifie l'équation (\ref{elliptic}) pour tout $(x,y) \in \Omega_{R}$.
    \begin{enumerate}
        \item Dessiner les ensembles $\Omega_{R}$ et $D_{R}$.
        \item On admet que la fonction $u$ admet un maximum en un point $M = (x_{0},y_{0}) \in D_{R}$. Supposons que $M \in \Omega_{R}$.
        \begin{enumerate}
            \item Quelle est la valeur de $\partial_{x}u(x_{0},y_{0})$ et de $\partial_{y}u(x_{0},y_{0})$?
            \item Quel est le signe de $\partial_{x}^{2}u(x_{0},y_{0})$ et de $\partial_{y}^{2}u(x_{0},y_{0})$?
            \item Montrer que cela n'est pas possible ({\it indication: utiliser (\ref{elliptic})}).
        \end{enumerate}
        \item \textit{[Principe du maximum]} Où se trouve alors le point $M$ ?
    \end{enumerate}
\end{exo}

% -------------------------------------------------
\begin{exo}
    On considère le problème suivant:
    \begin{eqnarray}
        y\,\partial_xu(x,y)+\partial_yu(x,y)&=&x\label{eqedp}\\
        u(x,0)&=&x\label{ciedp}
    \end{eqnarray}
    On considère les fonctions
    $$
        u_1(x,y)=xy-\frac{y^3}{3}+x-\frac{y^2}{2}\quad\mathrm{et}\quad u_2(x,y)=xy-\frac{y^3}{3}+e^{x-\frac{y^2}{2}}
    $$
    \begin{enumerate}
        \item Montrer que les fonctions $u_1$ et $u_2$ vérifient l'équation (\ref{eqedp}) pour tous les couples $(x,y)\in\R^2$.
        \item Laquelle des deux fonctions $u_1$, $u_2$ vérifie à la fois (\ref{eqedp}) et (\ref{ciedp}) pour tous les couples $(x,y)\in\R^2$?
        \item Déterminer les points critiques de la fonction $u_1$ et déterminer leur nature.
    \end{enumerate}
\end{exo}


% =================================================
\subsection{Équations aux dérivées partielles}
% =================================================


% -------------------------------------------------
\begin{exo} Les questions suivantes sont indépendantes.
    \begin{enumerate}
        \item Déterminer le domaine de définition $\mathcal{D}_{u} \subset \R ^{2}$ de la fonction $u(x,y) = \sin(x^{2} + \frac{1}{y})$ et
        montrer que $u$ est solution de l'équation
        $$
            \partial_{x}u(x,y) + 2xy^{2}\partial_{y}u(x,y) = 0
        $$
        pour tout $(x,y) \in \mathcal{D}_{u}$.
        \item Montrer que la fonction $u(x,y)=\ln(x+\exp(y/x))$ est bien définie pour tout
        $(x,y) \in \R _{+}^{\ast}\times \R _{+}^{\ast}$ et satisfait l'équation
        $$
            x\partial_{x}u(x,y) + y\partial_{y}u(x,y) = xe^{-u(x,y)}
        $$
        pour tout  $(x,y) \in \R _{+}^{\ast}\times \R _{+}^{\ast}$.
        \item
            \begin{enumerate}
                \item Soit $a \in \R ^{\ast}$. Etudier la fonction $f(t)=a+1 - e^{-at}$, puis montrer que la fonction
                    $$
                        u(x,t) = \frac{axe^{-at}}{a+1 - e^{-at}}
                    $$
                    est bien définie sur $\R \times \R _{+}$
                \item  Montrer que $u(x,t)$ est solution du problème
                    $$
                        \partial_{t}u(x,t) + u(x,t)\partial_{x}u(x,t) + au(x,t) = 0 \ \ , \ \ u(x,0)=x
                    $$
                    pour tout $(x,t) \in \R \times \R _{+}$.
                \item Calculer $\lim_{t \rightarrow +\infty}u(x,t)$.
            \end{enumerate}
    \end{enumerate}
\end{exo}

% -------------------------------------------------
% New : Question 3 du exam-juin13.pdf (Pollution de la riviere)
% -------------------------------------------------
\begin{exo}
    On note $u(x,t)$ la concentration (en {mol/l}) d'un produit chimique à distance $x \in [0,\pi]$ du bord d'une rivière à l'instant $t \geq 0$. On suppose que $u(x,t)$ satisfait le problème suivant :
    \begin{equation}
        \left.
        \begin{aligned}
            \partial_{t}u(x,t) &= \partial_{x}^{2}u(x,t) -2\,\partial_{x}u(x,t) \qquad \text{ pour } t \geq 0 \qquad \\
            u(x,0)             &= 2\,e^{x}\sin(x)
        \end{aligned}
        \right\}
        \text{ pour } x \in [0,\pi] \label{edpriviere}
    \end{equation}
    \begin{enumerate}
        \item Calculer la dérivée de $f(x)=2e^{x}\sin(x)$ sur $[0,\pi]$ que l'on notera $f'(x)$. Montrer que $f'$ s'annule en $\frac{3\pi}{4}$.
        \textit{(Bon à savoir~: $\sin(\frac{3\pi}{4})=\frac{\sqrt{2}}{2}$ et $\cos(\frac{3\pi}{4})=-\frac{\sqrt{2}}{2}$.)}
    \end{enumerate}

        \noindent On admettra dans la suite que la fonction $f$ admet un unique maximum en $\frac{3\pi}{4}$ sur $[0,\pi]$.

    \begin{enumerate}\setcounter{enumi}{1}
        \item Déterminer des constantes $a,b \in \mathbb{R}$ telles que la fonction
        \begin{equation*}
            u(x,t) = b\,e^{at}\,e^{x}\,\sin(x)
        \end{equation*}
        soit solution du problème (\ref{edpriviere}).

        \item On fixe $t$. Montrer que la fonction $x \mapsto u(x,t)$ admet un unique maximum (appelé pic de pollution) sur $[0,\pi]$ dont on précisera la valeur en fonction de $t$. Calculer en particulier la valeur du pic de pollution initial en $t=0$.
    \end{enumerate}
\end{exo}

% -------------------------------------------------
\begin{exo}
    On considère une tige métallique représentée par l'intervalle $[0,1]$ plongée dans un environnement de
    sorte qu'en tout point $x \in [0,1]$ la température (en degrés Celsius) de la tige est donnée par
    $$
        f(x) = \sin(\pi x).
    $$
    \begin{enumerate}
        \item Quelle est la température de la tige aux points $0$ et $1$ ? La température est-elle positive pour tout $x \in [0,1]$ ?\\

        On retire la tige de son environnement à un instant $t=0$ en conservant par un système de chauffage et de
        refroidissement les extrémités $0$ et $1$ de la tige à température constante égale à $0^{o}\mathrm{C}$. On note
        $u(x,t)$ la température à l'instant $t \geq 0$ d'un point $x \in [0,1]$ de la tige.\\
        Les équations de conservation de la physique impliquent que la vitesse d'accroissement $\partial_{t}u(x,t)$ de la
        température en temps est proportionnelle à la concavité $\partial_{x}^{2}u(x,t)$ en espace, pour tout $t \geq 0$, pour tout
        $x \in [0,1]$.\\
        Il existe donc $\alpha > 0$ tel que la température $u(x,t)$ vérifie
        \begin{equation}
            \partial_{t}u(x,t) = \alpha \partial_{x}^{2}u(x,t). \label{chaleur}
        \end{equation}
        Dans cette {\it équation de la chaleur}, on a la condition initiale
        \begin{equation}
            u(x,0) = f(x), \label{c_init}
        \end{equation}
        pour tout $0 \leq x \leq 1$, et les conditions au bord
        \begin{equation}
            u(0,t) = 0 \ \ , \ \ u(1,t) = 0, \label{c_bord}
        \end{equation}
        pour tout $t \geq 0$, tout $0 \leq x \leq 1$.
        \item Montrer que la fonction $f$ est solution du problème
        \begin{eqnarray}
            & & y''(x) = - \pi^2 y(x) \quad \forall x \in [0,1] \label{diffX}\\
            & & y(0)=y(1)=0 \label{ciX}
        \end{eqnarray}
        \item Montrer que la fonction $g(t)=e^{-\alpha\pi^2t}$ est solution du problème
        \begin{eqnarray}
            & & y'(t) = -\alpha \pi^2 y(t) \quad \forall t \geq 0 \label{diffT}\\
            & & y(0)=1 \label{ciT}
        \end{eqnarray}
        \item En utilisant les équations (\ref{diffX}) et (\ref{diffT}) vérifiées respectivement par $f$ et $g$, montrer que la fonction $u(x,t) = f(x)g(t)$ est solution de (\ref{chaleur}). Les conditions (\ref{c_init}-\ref{c_bord}) sont-elles satisfaites?
        \item A l'aide de la question précédente, déterminer le comportement de la température de la tige en tout point $x \in [0,1]$ pour un temps assez long.
    \end{enumerate}
\end{exo}

% -------------------------------------------------
\begin{exo}
    La réaction chimique de Belousov et Zhabotinskii fait intervenir plusieurs composants dont l'acide bromique $HB_rO_2$.\\
    À l'aide d'une sonde, on mesure la concentration de $HB_rO_2$, $u(x,t)$, à hauteur $x\ge 0$ dans le liquide à l'instant $t\ge 0$.
    Les réactions entre les différents composants impliquent que $u(x,t)$ satisfait une équation du type
    \begin{equation}
        \partial_{t}u(x,t)=\partial_{x}^{2}u(x,t)+f(u(x,t)) \label{acide bromique}
    \end{equation}
    pour tout $x\ge 0$, où $f$ est une fonction $\R \rightarrow \R $.\\
    Il s'agit ici d'une version raffinée de {\it l'équation de la chaleur} étudiée dans les problèmes précédents.
    En plus du phénomène de diffusion, on a d'autres réactions chimiques qui peuvent freiner ou accélérer l'évolution. Celles-ci peuvent être complexes, mais on suppose qu'elles ne dépendent que de la concentration de $HB_rO_2$ (et pas de sa variation ou du temps par exemple). On les modélise donc par le terme $f(u)$.
    \begin{enumerate}
        \item Soit $c>0$. Supposons qu'il existe une fonction $V(y): \R \rightarrow \R $ solution de l'équation différentielle
        \begin{equation}
            V''(y)-cV'(y)+f(V(y))=0 \label{acide bromique_2}
        \end{equation}
        On pose $u(x,t)=V(x+ct)$. Montrer alors que $u(x,t)$ est solution de l'équation (\ref{acide bromique}).
        \item Supposons que $f$ soit de la forme $f(u)=-au$ où $a>0$. Réécrire les équations (\ref{acide bromique}) et (\ref{acide bromique_2}).
        \begin{enumerate}
            \item Soit $g(y)=e^{ry}$. A quelle condition sur la constante $r\in\R$ la fonction $g$ est-elle solution de l'équation différentielle (\ref{acide bromique_2})?
            \item On posera dans la suite $r_1=(c-\sqrt{c^2+4a})/2$ et $r_2=(c+\sqrt{c^2+4a})/2$. Quelles sont les signes de $r_{1}$ et $r_{2}$ ?
            \item En déduire que la fonction définie par $V(y)=A_1e^{r_1y}+A_2e^{r_2y}$ est solution de (\ref{acide bromique_2}) pour tout choix de $A_1,A_2\in\R$.
        \end{enumerate}
        \item On suppose que la concentration $u(x,t)$ est donnée par $V(x+ct)$ (voir question a) ). En particulier, la fonction $V$ doit rester bornée et positive. Quelle condition doivent vérifier les constantes $A_1$ et $A_2$ pour cela? En déduire une expression de $u(x,t)$ vérifiant (\ref{acide bromique}).
        \item A partir de la question précédente, déterminer le comportement de la concentration à hauteur $x\ge 0$ fixée  pour un temps assez long.
    \end{enumerate}
\end{exo}


% ~~~~~~~~~~~~~~~~~~~~~~~~~~~~~~~~~~~~~~~~~~~~~~~~~
\finchapitre

