 %%%%%%%%%%%%%%%%%% PREAMBULE %%%%%%%%%%%%%%%%%%

\documentclass[12pt,a4paper]{article}

%\usepackage[francais]{exo7qcm}
\usepackage[francais,nosolutions]{exo7qcm}

\begin{document}
 
 
%%%%%%%%%%%%%%%%%% ENTETE %%%%%%%%%%%%%%%%%%

\LogoExoSept{2}

%\kern-2em
\hfill\textbf{Ann\'ee 2020}

\vspace*{0.5ex}
\hrule\vspace*{1.5ex} 
\hfil\textsc{\textbf{\Large QCM --- Mathématiques pour la SVT}}
\vspace*{1ex} \hrule 
\vspace*{5ex} 

\section{Questions -- Fiche 8 - Primitives}

\begin{question}
Quelle fonction est une primitive de $f(x) = \frac{6x+2}{3x^2+2x-1}$ ?
\begin{answers}
    \bad{$F(x) = \frac{1}{3x^2+2x-1}$}

    \bad{$F(x) = \frac{3x^2+2x-1}{6x+2}$}

    \good{$F(x) = \ln(3x^2+2x-1)$}

    \bad{$F(x) = \ln\left( \frac{6x+2}{3x^2+2x-1}\right)$}
\end{answers}

\begin{explanations}
La dérivée de $F(x) = \ln(3x^2+2x-1)$ est $\frac{u'}{u}$ avec $u=3x^2+2x-1$, 
donc $F'(x) = \frac{6x+2}{3x^2+2x-1} = f(x)$. 
\end{explanations}

\end{question}


\begin{question}
Combien vaut l'intégrale $\int_1^3 x \, dx$ ?
\begin{answers}
    \good{$4$}

    \bad{$5$}

    \bad{$8$}

    \bad{$10$}
\end{answers}

\begin{explanations}
Un primitive de $f(x) = x$ est $F(x) = \frac12x^2$.
Ainsi $\int_1^3 x \, dx 
= \left[ \frac12x^2 \right]_1^3 
= \frac12\left( 3^2-1^2 \right) 
= \frac82 = 4$.
\end{explanations}

\end{question}


\begin{question}
Combien vaut l'intégrale $\int_1^2 1+\frac1x+\frac1{x^2} \, dx$ ?
\begin{answers}
    \bad{$\frac12+\ln(2)$}

    \good{$\frac32+\ln(2)$}

    \bad{$-\frac12+\ln(2)$}

    \bad{$\frac12-\ln(2)$}
\end{answers}

\begin{explanations}
Un primitive de $f(x) = 1+\frac1x+\frac1{x^2}$ est $F(x) = x+\ln(x)-\frac1x$.
Ainsi $\int_1^2 f(x) \, dx 
= \left[ F(x) \right]_1^2
= F(2)-F(1) 
= \left( 2+\ln(2)-\frac12 \right) -  \left( 1+\ln(1)-\frac11 \right)
= \frac32+\ln(2)
$.
\end{explanations}

\end{question}


\begin{question}
On considère $f(x) = \frac{1}{x^2-x}$. On écrit
$\frac{1}{x^2-x} = \frac{A}{x} + \frac{B}{x-1}$.
\begin{answers}
    \good{On a $A=-1$ et $B=1$.}

    \bad{On a $A=1$ et $B=-1$.}

    \bad{Une primitive de $f$ est $F(x) = -\frac{A}{x^2} -\frac{B}{(x-1)^2}$}

    \good{Une primitive de $f$ est $F(x) = A \ln(x) + B\ln(x-1)$}
\end{answers}

\begin{explanations}
$f(x) = \frac{1}{x^2-x} = \frac{-1}{x} + \frac{1}{x-1}$ donc $A=-1$ et $B=1$.
Une primitive de $f$ est $F(x) = A \ln(x) + B\ln(x-1)$.
\emph{Bonus.} On a donc  $F(x) = - \ln(x) + \ln(x-1) = \ln\left( \frac{x-1}{x} \right)$.
\end{explanations}

\end{question}


\begin{question}
Combien vaut l'intégrale $\int_0^2 xe^{x^2+1} \, dx$ ?
\begin{answers}
    \bad{$e\left( e^4+1 \right)$}

    \bad{$e\left( e^4-1 \right)$}

    \bad{$\frac12e\left( e^4+1 \right)$}

    \good{$\frac12e\left( e^4-1 \right)$}
\end{answers}

\begin{explanations}
Une primitive de $f(x) = xe^{x^2+1}$ est $F(x) = \frac12e^{x^2+1}$.
Ainsi $\int_0^2 f(x) \, dx 
= \left[ F(x) \right]_0^2
= F(2)-F(0) 
= \frac12\left( e^5-e^1 \right)
= \frac12e\left( e^4-1 \right)
$.
\end{explanations}

\end{question}



\end{document}
