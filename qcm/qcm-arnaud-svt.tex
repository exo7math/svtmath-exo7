 %%%%%%%%%%%%%%%%%% PREAMBULE %%%%%%%%%%%%%%%%%%

\documentclass[12pt,a4paper]{article}

\usepackage[francais]{exo7qcm}
%\usepackage[francais,nosolutions]{exo7qcm}

\begin{document}
 
 
%%%%%%%%%%%%%%%%%% ENTETE %%%%%%%%%%%%%%%%%%

\LogoExoSept{2}

%\kern-2em
\hfill\textbf{Ann\'ee 2020}

\vspace*{0.5ex}
\hrule\vspace*{1.5ex} 
\hfil\textsc{\textbf{\Large QCM de mathématiques}}
\vspace*{1ex} \hrule 
\vspace*{5ex} 

\section{Questions -- Fonctions usuelles -- Facile}




\begin{question}
Quel est le domaine de définition de la fonction $\sqrt{x+1}$ ?
\begin{answers} 
    \bad{$\Rr \setminus \{1\}$}
    \bad{$]1,+\infty[$}
    \good{$[-1,+\infty[$}
    \bad{$]-1,+\infty[$}
\end{answers}
\end{question}



\begin{question}
Soit $f(x) = x^2+1$.
\begin{answers} 
    \good{Le domaine de définition de $f$ est $\Rr$}
    \bad{Le domaine de définition de $f$ est $[1,+\infty[$}
    \good{Pour tout $x\in\Rr$, $f(-x)=f(x)$}
    \bad{Pour tout $x\neq0$, $f(x)\ge2$}
\end{answers}
\end{question}


\begin{question}
Soit $f(x) = e^{-x}+1$.
\begin{answers} 
    \bad{$\lim_{x\to+\infty} f(x) = -\infty$}
    \bad{$\lim_{x\to+\infty} f(x) = 0$}
    \bad{$\lim_{x\to-\infty} f(x) = 1$}
    \good{$\lim_{x\to-\infty} f(x) = +\infty$}
\end{answers}
\end{question}


\begin{question}
Quelles sont les égalités vraies ?
\begin{answers} 
    \bad{$\cos(\frac\pi6) = \frac{\sqrt{3}}{\sqrt{2}}$}
    \bad{$\tan(\frac\pi2) = 0$}
    \good{$\sin^2 (\frac{7\pi}{3}) = \frac34$}
    \good{$\tan'(\frac\pi4) = 2$}
\end{answers}
\end{question}


\section{Questions -- Fonctions usuelles -- Moyen}

\begin{question}
Quel est le domaine de définition de la fonction $\ln\left(\frac{x-1}{x}\right)$ ?
\begin{answers} 
    \bad{$\Rr \setminus\{0\}$}
    \bad{$\Rr \setminus\{0,1\}$}
    \good{$]0,1[$}
    \bad{$]-\infty,0[ \cup ]1,+\infty[$}
\end{answers}
\end{question}


\begin{question}
Quelles sont les égalités vraies ?
\begin{answers} 
    \good{$\ln(a^2b^3) = 2\ln(a)+3\ln(b)$}
    \bad{$\ln(\frac{1}{\sqrt{x}} = \frac12\ln(x)$}
    \good{$e^{\ln(2)+\ln(3)} = 6$}
    \bad{$e^{\frac12\ln3} = \sqrt[3]{2}$}
\end{answers}
\end{question}


\section{Question 1 (avec variantes) -- Intégrale simple -- Facile}


\begin{question}
\qid{svt-integrale!1}

Combien vaut l'intégrale $\int_0^1 \exp(3x)\,dx$ ?

% code sage : integrate(exp(3*x),x,0,1)

\begin{answers}  
    \good{$\frac13(e^3-1)$}
    \bad{$\frac13e^3$}
    \bad{$\frac13(e^2-\frac13)$}
    \bad{$\frac13(e-\frac13)$}
\end{answers}
\end{question}


\begin{question}
\qid{svt-integrale!2}

Combien vaut l'intégrale $\int_0^{\frac\pi2} \sin(2x)\,dx$ ?

% code sage : integrate(sin(2*x),x,0,pi/2)

\begin{answers}  
    \good{$1$}
    \bad{$0$}
    \bad{$\pi$}
    \bad{$\frac\pi2$}
\end{answers}
\end{question}


\begin{question}
\qid{svt-integrale!3}

Combien vaut l'intégrale $\int_0^{\frac\pi2} \cos(3x)\,dx$ ?

% code sage : integrate(cos(3*x),x,0,pi/2)

\begin{answers}  
    \good{$\frac13$}
    \bad{$0$}
    \bad{$\pi$}
    \bad{$\frac\pi3$}
\end{answers}
\end{question}


\begin{question}
\qid{svt-integrale!4}

Combien vaut l'intégrale $\int_0^2 \exp(2x)\,dx$ ?

% code sage : integrate(exp(2*x),x,0,2)

\begin{answers}  
    \good{$\frac12(e^4-1)$}
    \bad{$\frac12(e^3-1)$}
    \bad{$\frac12(e^2-1)$}
    \bad{$\frac12(e-1)$}
\end{answers}
\end{question}



\section{Question 2 (variantes à finir) -- Points critiques -- Facile}

\begin{question}
\qid{svt-critique!1}

Quels sont les points critiques de $f(x,y) = x^4+2x^2+(y-1)^3.$


\begin{answers}  
    \good{$(0,1)$ et $(-1,1)$}
    \bad{$(0,-1)$ et $(1,0)$}
    \bad{$(0,0)$ et $(1,1)$}
    \bad{$(0,0)$ et $(1,-1)$}    
\end{answers}
\end{question}



\section{Question 3 (variantes à finir) -- Type de points critiques -- Facile}


\begin{question}
\qid{svt-type!1}
Soit $f(x,y) = f(x,y) = x^2 - y^2 - 2x + 4y - 1.$
Quel est le type du point critique  en $(1,2)$ ?

% var('x','y')
% expand( (x-1)^2-(y-2)^2 + 2 )

\begin{answers}  
    \good{un point-selle}
    \bad{un minimum local}
    \bad{un maximum local}
\end{answers}
\end{question}


\section{Question 4 (variantes à finir) -- Equations différentielles -- Facile}

\begin{question}
\qid{svt-eqdiff!1}

Quelle est la solution de l'équation différentielle $y'(x)=x^2 y(x)$ qui vérifie $y(0)=1$ ?

%x = var('x')
%y = function('y')(x)
%yy = diff(y,x)
%f = desolve(yy == x^2*y, y, ics=[0,1])
%f = e^(1/3*x^3)


\begin{answers}  
    \good{$y(x) = \exp(\frac13x^3)$}
    \bad{$y(x) = \exp(\frac12x^)$}
    \bad{$y(x) = \exp(-x)$}
    \bad{$y(x) = \exp(x)$}
\end{answers}
\end{question}

\end{document}
