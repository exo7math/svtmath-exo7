\documentclass[11pt,class=report,crop=false]{standalone}
\usepackage{exo7sv}

\begin{document}

%%%%%%%%%%%%%%%%%%%%%%%%%%%%%%%%%%%%%%%%%%%%%%%%%%%%%%%%%%%%%%%%%%%%%%
%%%%%%%%%%%%%%%%%%%%%%%%%%%%%%%%%%%%%%%%%%%%%%%%%%%%%%%%%%%%%%%%%%%%%%

\section*{Résolution d'équations différentielles au moyen des primitives}

\setcounterexo{32}

% exercice 33
\exercice{}
\enonce
Reprenons le problème déjà rencontré dans la première partie :
 \begin{eqnarray}
            x'(t) & = & (\alpha - 2t -1)x(t) \label{ncdiffx} \\
            x(0)  & = & x_0 > 0 \label{cix}
        \end{eqnarray}

\begin{enumerate}
\item Quelle est la primitive (à constante près) de $\frac{x'(t)}{x(t)}$ et celle de $\alpha -2t -1$ ? Déduire que si $x(t)$ est solution de l'équation différentielle,
on a $\ln| x| = \alpha t - t^2 -t + C$.
\item Déterminer la constante $C$ en fonction de la valeur initiale $x_0$.
\item Extraire la valeur de $x$ comme fonction de $t$.
\end{enumerate}
\finenonce

\indication
Si deux fonctions ont la même dérivée, alors les fonctions sont égales à une constante près :
si $f'(x)=g'(x)$ (pour tout $x$) alors $f(x)=g(x)+C$ (pour tout $x$), pour une certaines constante réelle $C$.

On peut admettre ici que la solution cherchée est partout positive, ainsi $|x(t)|=x(t)$.
\finindication

\correction

\video{eB8qY29jzA0}

\sauteligne
\begin{enumerate}
  \item 
Les primitives de $\frac{x'(t)}{x(t)}$ sont les $\ln ( |x(t)| ) + C_1$.
Les primitives de $\alpha-2t-1$ sont les $\alpha t-t^2-t+C_2$.
Par l'équation différentielle $\frac{x'(t)}{x(t)} = \alpha - 2t -1$, donc les primitives sont égales (à une constante près) :
$$\ln ( |x(t)| ) = \alpha t-t^2-t+C$$


  \item En $t=0$ on a $x(t)=x_0>0$ donc on obtient $\ln (x_0) = C$.

  \item Reprenons l'égalité $\ln ( |x(t)| ) = \alpha t-t^2-t+C$,
alors
\begin{align*}
|x(t)| 
  &= e^{\alpha t-t^2-t+C} \\
  &= e^{(\alpha-1) t-t^2} \cdot e^{C} \\
  &= x_0 \cdot e^{(\alpha-1) t-t^2} \qquad \text{ car } e^C = e^{\ln (x_0)} = x_0 \\
\end{align*}

On prouve pour finir que $x(t)>0$. En effet la solution nulle est solution de l'équation différentielle ; comme deux courbes intégrales ne se coupent pas, notre solution ne s'annule pas et comme $x(0)>0$ alors $x(t)$ reste positive.

Conclusion $|x(t)| = x(t)$ et donc $x(t) = x_0 e^{(\alpha-1) t-t^2}$.

\end{enumerate}
\fincorrection
\finexercice

% exercice 34
\exercice{}
\enonce
Reprenons le problème déjà rencontré dans la première partie :
\begin{eqnarray}
y'(t) & = & - K \, y(t) \, (1-y(t)) \label{cix1} \\
y(0)  & = & y_0  \text{ avec } 0 < y_0 < 1 \label{cix2}
\end{eqnarray}

\begin{enumerate}
	\item Calculer la primitive (à constante près) de $$\frac{y'(t)}{y(t)(1-y(t))}.$$ 
	Déduire que si $y(t)$ est solution  de l'équation différentielle,
	on a $$\ln \left|\frac{y}{1-y}\right| = -Kt + C.$$
	\item Déterminer la constante $C$ en fonction de la valeur initiale $y_0$.
	\item Extraire la valeur de $y$ comme fonction de $t$.
\end{enumerate}  
\finenonce

\indication
\sauteligne
\begin{itemize}
  \item Trouver des constantes $ A, B $ telle que 
	$\frac{1}{y(1-y)} = \frac{A}{y} + \frac{B}{1 - y}$.
  \item Récrire l'équation différentielle sous la forme $\frac{y'(t)}{y(t)(1-y(t))}=-K$ et intégrer de chaque côté.
  \item Si $ F $ et $ G $ sont des primitives d'une fonction $ f $, alors 
$ F = G + C $, où $ C \in \Rr $ est une constante. 
  \item On admet ici que notre solution vérifie $0<y(t)<1$ pour tout $t$.
\end{itemize}
\finindication

\correction

\video{U7Im6uAOT7w}

\sauteligne
\begin{enumerate} 
	\item 
	Si $ F(t) $ est une primitive de $ f(t) $ et $ u(t) $ est une fonction 
	dérivable, alors $ F(u(t)) $ est une primitive de $ u'(t) f(u(t)) $. 
	On cherche des constantes $ A, B $ telle que 
	$ 
	\frac{1}{y(1-y)} = \frac{A}{y} + \frac{B}{1 - y} 
	$. Comme 
	\begin{equation*} 
		\frac{A}{y} + \frac{B}{1 - y} = \frac{A - Ay + By}{y(1-y)} 
		= \frac{(B - A)y + A}{y(1-y)} 
	\end{equation*} 
	il faut que $ (B - A)y + A = 1 $ 
	\begin{equation*} 
		\begin{cases} 
			B - A &= 0 \\ 
			A &= 1 
		\end{cases} 
		\quad \Longleftrightarrow \quad 
		\begin{cases} 
			A &= 1 \\ 
			B &= 1. 
		\end{cases} 
	\end{equation*} 
	Donc 
	\begin{equation*} 
		\begin{split} 
			\int \frac{y'(t)}{y(t)(1-y(t))} \, dt 
			&=  
			\int y'(t) \, \frac{1}{y(t)(1-y(t))} \, dt 
			= 
			\int y'(t) \left(\frac{1}{y(t)} + \frac{1}{1 - y(t)}\right) dt \\ 
			&= 
			\int \left(\frac{y'(t)}{y(t)} + \frac{y'(t)}{1 - y(t)}\right) dt 
			= 
			\int \frac{y'(t)}{y(t)} + \int \frac{y'(t)}{1 - y(t)} dt \\ 
			&=  
			\ln |y(t)| - \ln|1 - y(t)| + C \\ 
			&= 
			\ln \left|\frac{y(t)}{1 - y(t)}\right| + C. 
		\end{split} 
	\end{equation*} 
	On écrit l'équation différentielle (\ref{cix1}) sous la forme 
	\begin{equation*} 
		\frac{y'(t)}{y(t)(1-y(t))} = -K. 
	\end{equation*} 
	On sait déj\`a qu'une primitive de $ \frac{y'(t)}{y(t)(1-y(t))} $ 
	est la fonction $ \ln \left|\frac{y(t)}{1 - y(t)}\right| $. De plus, 
	$ -Kt $ est une primitive de $ -K $. Comme on sait que deux primitives 
	d'une même fonction ne différent que d'une constante, il existe 
	alors une constante $ C $ telle que 
	\begin{equation} \label{y} 
	\ln \left|\frac{y(t)}{1 - y(t)}\right| = -Kt + C. 
	\end{equation} 
	\item 
	Si on pose $ t = 0 $ dans (\ref{y}), alors 
	\begin{equation*} 
		C = \ln \left|\frac{y(0)}{1 - y(0)}\right| + K \cdot 0 
		= \ln \frac{y_0}{1 - y_0}
	\end{equation*} 
	\item 
	En remplace $ C $ par la valeur trouvée et obtient 
	\begin{equation*} 
		\ln \left|\frac{y(t)}{1 - y(t)}\right| = -Kt + \ln \frac{y_0}{1 - y_0} 
		\quad 
		\Longleftrightarrow 
		\quad 
		\ln \left|\frac{(1 - y_0)y(t) }{y_0(1 - y(t))}\right| = -Kt. 
	\end{equation*} 
\end{enumerate} 
En appliquant la fonction exponentielle et en admettant ici $0<y(t)<1$, on trouve :
\begin{equation*} 
	\begin{split} 
		\frac{(1 - y_0)y(t) }{y_0(1 - y(t))} = e^{-Kt} 
		& \Longleftrightarrow 
		 (1 - y_0)y(t) =  y_0 (1 - y(t))e^{-Kt} 
		\Longleftrightarrow 
		(1 - y_0 + y_0 e^{-Kt})y(t)  = y_0 e^{-Kt} \\ 
		& \Longleftrightarrow 
		y(t) = \frac{y_0 e^{-Kt}}{1 - y_0 + y_0 e^{-Kt}}. 
	\end{split} 
\end{equation*} 
\fincorrection
\finexercice

\end{document}