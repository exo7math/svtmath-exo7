\documentclass[11pt,class=report,crop=false]{standalone}
\usepackage{exo7sv}

\begin{document}

%%%%%%%%%%%%%%%%%%%%%%%%%%%%%%%%%%%%%%%%%%%%%%%%%%%%%%%%%%%%%%%%%%%%%%
%%%%%%%%%%%%%%%%%%%%%%%%%%%%%%%%%%%%%%%%%%%%%%%%%%%%%%%%%%%%%%%%%%%%%%

\entete{Université de Lille}{Mathématiques pour la SVT}

\titre{Fiche 8. \quad Primitives} 

\encadre{
	\emph{Savoir.}
	\begin{itemize}[label=$\square$]
		\item Connaître la définition d'une primitive.
        \item Connaître les formules des primitives usuelles.
	\end{itemize}
	\emph{Savoir-faire.}
	\begin{itemize}[label=$\square$]
		\item Savoir déterminer une primitive.
		\item Savoir utiliser les primitives pour calculer des intégrales.
	\end{itemize}
}


\insertvideo{QACseEPX0Ds}{Fiche 8. Primitives}

\bigskip

\subsection*{Primitives}


\begin{itemize}
  \item \textbf{Définition.} Soit $f : I \to \Rr$ une fonction.
On dit qu'une fonction $F$ est une \textbf{primitive} de $f$ si pour tout $x\in I$ :
\mybox{$F'(x)=f(x)$} 

% (Il est sous-entendu que $F$ doit être dérivable.)

  \item Exemples.
\begin{itemize}
	\item $F(x)=\displaystyle\frac{x^3}{3}$ est une primitive de $f(x)=x^2$.
	\item $\ln(x)$ est une primitive de $\frac1x$ sur $]0,+\infty[$.
  %  \item $\ln(x) + 1$ est aussi une primitive de de $\frac1x$ sur $]0,+\infty[$.
\end{itemize}

  \item Trouver une primitive est l'opération inverse du calcul de la dérivée.

  \item Exercice. Trouver une primitive de chacune des fonctions suivantes :
\begin{multicols}{2}
\begin{itemize}
	\item[$\bullet$] $x$
    \item[$\bullet$] $\cos(x)$
    \item[$\bullet$] $\sin(x)$
    \item[$\bullet$] $e^{-x}$
    \item[$\bullet$] $\dfrac3x - \dfrac{7}{x^2} + 1$
    \item[$\bullet$] $\frac{1}{\cos^2(x)}$
\end{itemize}
\end{multicols}
\end{itemize}


\subsection*{Calculs d'intégrales à l'aide d'une primitive}


\textbf{Théorème.}
Soit $f$ une fonction définie sur un intervalle $[a,b]$.
Soit $F$ une primitive de $f$. Alors:
\mybox{
$\displaystyle \int_{a}^{b}f(x)\;\dd x=F(b)-F(a)$
}

\begin{itemize}
  \item C'est le moyen le plus efficace pour calculer des intégrales ! 

  \item Notation par des crochets. 
\myboxinline{$\big[F(x)\big]^b_a = F(b)-F(a)$}.

  
  \item Exemple.
  \[\int_{1}^{2}x^2\;\dd x=\bigg[\frac{x^3}{3}\bigg]^2_1=\frac{2^3}{3}-\frac{1^3}{3} = \frac{7}{3}.\]

  \item Exemple. \[\int_{2}^{7}\frac1x \;\dd x =\big[ \ln(x) \big]_2^7= \ln(7) - \ln(2) = \ln\left(\tfrac72\right).\]
\end{itemize}


\subsection*{Toutes les primitives}

\begin{itemize}
   \item Une primitive n'est pas unique ! Soit $f(x)=x^2$, alors $F(x)=\displaystyle\frac{x^3}{3}$ est une primitive. Mais la fonction $G(x) = \displaystyle\frac{x^3}{3} + 2$ est aussi une primitive (dérivez-la pour vérifier). Il y a donc plusieurs primitives. En fait toutes les fonctions $\displaystyle\frac{x^3}{3} + c$, où $c$ est une constante, sont des primitives. Nous généralisons ceci à toutes les fonctions :

 % \item 
  \mybox{\textbf{Proposition.} Si $F(x)$ est une primitive de $f(x)$, alors les autres primitives sont de la forme $F(x)+c$ où $c\in\Rr$ est une constante.}

  \item Exemple. Les primitives de $x^4-3x+5$ sont les fonctions
$\frac{1}{5}x^5 - \frac{3}{2}x^2+5x + c$, où $c\in\Rr$ est une constante.

  \item Exercice. Vérifier que les primitives de la fonction $\sqrt{x}=x^{\frac12}$ sont les fonctions $\frac23 x\sqrt{x} +c = \frac{2}{3} x^{\frac32} + c$.

  \item Pour calculer une intégrale, vous choisissez la primitive que vous voulez car $\big[F(x)\big]_a^b$ donne le même résultat quelle que soit la primitive.
\end{itemize}




\subsection*{Primitives usuelles}


\subsubsection*{Primitives des fonctions classiques}

Ici $c$ désigne une constante réelle.

\begin{center}
	\begin{tabular}[t]{|c|c@{\vrule depth 1.2ex height 3ex width 0mm \ }|}
		\hline
		\textbf{Fonction}         & \textbf{Primitives} \\ \hline
		$x^n$          & $\frac{x^{n+1}}{n+1}+c$  \quad ($n \in \Nn$)   \\ \hline
%		$x^{-n}$         & $\frac{x^{1-n}}{1-n}+c$  \quad ($n \in \Nn\setminus\{0,1\}$, $c\in\Rr$)   \\ \hline
		$\frac{1}{x}$  & $\ln(|x|)+c$              \\ \hline
		$x^\alpha$     & $\frac{x^{\alpha+1}}{\alpha+1}+c$  \quad ($\alpha \in \Rr\setminus\{-1\}$)   \\ \hline
		$\sqrt{x}=x^{\frac12}$    & $\frac23x\sqrt{x}+c=\frac23x^{\frac32}+c$  \quad (c'est $\alpha=\frac12$)   \\ \hline
		$e^x$          & $e^x+c$                        \\ \hline
		$\cos(x)$      & $\sin(x)+c$                \\ \hline
		$\sin(x)$      & $-\cos(x)+c$                     \\ \hline
%		$\tan(x)$      & $-\ln(|\cos(x)|)+c$  \quad ($c\in\Rr$)            \\ \hline
%		$\frac{1}{1+x^2}$      & $\arctan(x)+c$  \quad ($c\in\Rr$)            \\ \hline
	\end{tabular}
\end{center}

Ces formules sont à maîtriser ! Mais ce sont juste les formules des dérivées que vous connaissez déjà.

\subsubsection*{Primitives pour une composition}

Ici $u$ est une fonction qui dépend de $x$ ; $c$ désigne une constante réelle.

\begin{center}
	\begin{tabular}[t]{|c|c@{\vrule depth 1.2ex height 3ex width 0mm \ }|}
		\hline
		\textbf{Fonction}         & \textbf{Primitive} \\ \hline
		$u'u^n$         & $\frac{u^{n+1}}{n+1}+c$  \quad ($n \in \Nn$)   \\ \hline
%		$u'u^{-n}$         & $\frac{u^{1-n}}{1-n}+c$  \quad ($n \in \Nn\setminus\{0,1\}$, $c\in\Rr$)   \\ \hline
		$\frac{u'}{u}$    & $\ln(|u|)+c$         \\ \hline
		$u'u^\alpha$         & $\frac{u^{\alpha+1}}{\alpha+1}+c$  \quad ($\alpha \in \Rr\setminus\{-1\}$)   \\ \hline
%		$\frac{u'}{2\sqrt{u}}$    & $\sqrt{u}+c$  \quad ($c\in\Rr$)    \\ \hline
		$u'e^u$         & $e^u+c$                     \\ \hline
		$u'\cos(u)$      & $\sin(u)+c$                      \\ \hline
		$u'\sin(u)$      & $-\cos(u)+c$                        \\ \hline
	%	$u'\tan(u)$      & $-\ln(|\cos(u)|)+c$  \quad ($c\in\Rr$)            \\ \hline
	%	$\frac{u'}{1+u^2}$      & $\arctan(u)+c$  \quad ($c\in\Rr$)            \\ \hline
	\end{tabular}
\end{center}


\begin{itemize}
   \item Exemple.
   Comment calculer $\int_1^2 x e^{x^2} \dd x$ ? 
   Avec $u(x) = x^2$ (et donc $u'(x)=2x$) on a 
   $2x e^{x^2}  = u'(x)e^{u(x)}$ dont une primitive est $e^{x^2} = e^{u(x)}$.
  Ainsi 
  $$\int_1^2 x e^{x^2} \dd x = \tfrac12\int_1^2 2x e^{x^2} \dd x
  = \tfrac{1}{2}\left[ e^{x^2} \right]_1^2 = \tfrac{1}{2}(e^{4}-e).$$

   \item Exemple.
  On sait que $\tan(x)=\frac{\sin(x)}{\cos(x)}=-\frac{\cos'(x)}{\cos(x)}$. Par le tableau précédent, les primitives de la fonction tangente sont les fonctions de la forme $F(x)=-\ln(|\cos(x)|)+c$ où $c$ est une constante réelle.

\end{itemize}




\end{document}