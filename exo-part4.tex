\documentclass[11pt,class=report,crop=false]{standalone}
\usepackage{exo7sv}

\begin{document}

%%%%%%%%%%%%%%%%%%%%%%%%%%%%%%%%%%%%%%%%%%%%%%%%%%%%%%%%%%%%%%%%%%%%%%
%%%%%%%%%%%%%%%%%%%%%%%%%%%%%%%%%%%%%%%%%%%%%%%%%%%%%%%%%%%%%%%%%%%%%%

\section*{Calcul d'intégrales}


\setcounterexo{19}

% exercice 20
\exercice{}
\enonce
Calculer les intégrales suivantes~:
    \begin{examplescol}{4}
        \item $\int \limits _{2} ^{4} (x-2)^{5}\,dx$
        \item $\int \limits _{0} ^{1} \sqrt{3y+1}\, dy$
        \item $\int \limits _{0} ^{\frac{\pi}{4}} 5\sin(2\theta) \,d\theta$
        \item $\int \limits _{0} ^{1} \frac{1}{(2t+1)^{3}} dt$
    \end{examplescol}
\finenonce

\indication
Il s'agit de faire des changements de variable simples :
\begin{enumerate}
	\item Poser $u= x-2$.
	\item Poser $u=3y+1$.
	\item Poser $u = 2\theta$.
	\item Poser $u = 2t+1$. 
\end{enumerate}
\finindication

\correction

\video{07JZKypzCrU}

\sauteligne
\begin{enumerate}
  \item $I = \int \limits _{2} ^{4} (x-2)^{5}\,dx$
  
  On pose $u= x-2$. On alors $du = dx$.
  La calcul de l'intégrale va de la borne $x=2$ à $x=4$.
  Ce qui donne comme bornes en $u$ : de $u=2-2=0$ à $u=4-2$ (car $u=x-2$).

  La formule de changement de variable transforme l'intégrale en $x$ en une intégrales en $u$.
  On n'oublie pas de changer les bornes (les bornes en $x$ sont remplacées par des bornes en $u$) et aussi l'élément différentiel (la nouvelle intégrale a pour élément différentiel $du$).
  
  $$I = \int \limits _{x=2} ^{x=4} (x-2)^{5}\,dx
  = \int \limits _{u=0} ^{u=2} u^{5}\,du$$
  
  Cette nouvelle intégrale est beaucoup plus facile à calculer :
  $$I = \int \limits _{u=0} ^{u=2} u^{5}\,du = \left[ \frac{u^6}{6} \right]_{u=0} ^{u=2} = \frac{2^6}{6} = \frac{64}{6} = \frac{32}{3}.$$
  

  \item $\int \limits _{0} ^{1} \sqrt{3y+1}\, dy$
  
  On pose $u= 3y+1$. On alors $du = 3\,dy$ et donc $dy = \frac{du}{3}$.
  Les bornes en $x$ sont de $x=0$ à $x=1$ et deviennent en la variable $u$ : de $u=1$ à $u=4$.
  
  \begin{align*}
  	\int \limits _{y=0} ^{y=1} \sqrt{3y+1}\, dy
   &= \int \limits _{u=1} ^{u=4} \sqrt{u}\,\frac{du}{3} \\
   &= \frac13\int \limits _{u=1} ^{u=4} u^{\frac12} \,du \\
   &= \frac13\left[ \frac23u^{\frac32} \right]_{u=1} ^{u=4} \qquad \text{ car une primitive de $u^\alpha$ est $\frac1{\alpha+1} u^{\alpha +1 }$ avec ici $\alpha=\frac12$} \\
   &= \frac13 \left(\frac23 \cdot 4^{\frac32}-\frac23 \cdot 1^{\frac32} \right) \\
   &= \frac13 \left(\frac23 \cdot 8-\frac23 \cdot 1^{\frac32} \right)\qquad \text{ car $4^{\frac32} = (4^3)^\frac12 = \sqrt{64} = 8$} \\
   &= \frac13 \cdot \frac23 \cdot 7 \\
   &= \frac{14}9
   \end{align*}

   \item $\int \limits _{0} ^{\frac{\pi}{4}} 5\sin(2\theta) \,d\theta$ 
   
   Posons $u = 2\theta$, alors $du = 2\,d\theta$ et donc $d\theta = \frac{du}{2}$.
   $\theta$ varie de $\theta=0$ à $\theta=\frac\pi4$ donc $u$ varie de $u=0$ à $u=\frac\pi2$.
   
   $$\int \limits _{\theta=0} ^{\theta=\frac{\pi}{4}} 5\sin(2\theta) \,d\theta
   = \int \limits _{u=0} ^{u=\frac{\pi}{2}} 5\sin(u) \,\frac{du}{2} 
   = \frac52 \big[-\cos(u) \big]_{u=0} ^{u=\frac{\pi}{2}} 
   = \frac52 \big( -\cos(\frac{\pi}{2}) - (-\cos(0)) \big)
   = \frac52 (0+1) = \frac52.$$
   
   \item $\int \limits _{0} ^{1} \frac{1}{(2t+1)^{3}} dt$   
 
   Posons $u = 2t+1$. Donc $du= 2 \,dt$ ou encore $dt = \frac{du}{2}$.
   $t$ varie de $t=0$ à $t=1$ donc $u$ varie de $u=1$ à $u=3$.
   $$\int \limits _{t=0} ^{t=1} \frac{1}{(2t+1)^{3}} dt
   = \int \limits _{u=1} ^{u=3} \frac{1}{u^{3}} \frac{du}{2}
   = \frac12 \int \limits _{u=1} ^{u=3} u^{-3} du
   = \frac12\left[ -\frac12u^{-2} \right]_{u=1} ^{u=3}
   = \frac14 \left( - \frac{1}{3^2} + \frac{1}{1^2} \right)
   = \frac14 \cdot \frac{8}{9}
   = \frac{2}{9}$$
   
   On a utilisé que $\frac{1}{u^2} = u^{-2}$, $\frac{1}{u^3} = u^{-3}$ et qu'une primitive de $u^\alpha$ est $\frac1{\alpha+1} u^{\alpha +1 }$, ce qui donne pour 
   $\alpha=-3$ : une primitive de  $u^{-3}$ est $-\frac12u^{-2}$.   
   Si vous préférez vous pouvez dire directement qu'une primitive de $\frac1{u^3}$ est $-\frac12\frac{1}{u^2}$.
    
\end{enumerate}
\fincorrection


% exercice 21
\exercice{}
\enonce 
Calculer les intégrales suivantes~: 
\begin{examplescol}{2}
	\item $\int \limits _{0} ^{1} xe^{x^{2}}dx$
	\item $\int \limits _{0} ^{\frac{\pi}{2}} \sqrt{3\cos(y)}\sin{(y)} \,dy $ 
	\item $\int \limits _{0} ^{\frac{\pi}{4}} \tan(\theta) \,d\theta$
	\item $\int \limits _{1} ^{5} \frac{e^{\sqrt{x}}}{2\sqrt{x}}dx$
\end{examplescol}
\finenonce

\indication
Vous pouvez soit reconnaître une forme $u'f'(u)$ et utiliser qu'une primitive de la fonction $ u'(x) f'(u(x)) $ est $ f(u(x)) + C $, c'est la méthode de substitution, 
$ C \in \Rr $ ou bien vous pouvez faire un changement de variable. 
\finindication

\correction

\video{RUej6jMp39g}

   La méthode de substitution ou le changement de variable sont des méthodes équivalentes. Choisissez celle que vous préférez !
   
\begin{enumerate} 
	\item $\int  _{0} ^{1} xe^{x^{2}}dx$
	
   
	\textbf{Substitution.}
	On sait que $ (x^2)' = 2x $. Alors $ x e^{x^{2}} $ est de la forme 
	$ \tfrac{1}{2} u'(x) f'(u(x)) $ avec $ u(x) = x^2 $ et $ f(x) = 
	e^{x} $ et on obtient 
	\begin{equation*} 
		\int  _{0} ^{1} xe^{x^{2}}dx = \left[\frac{1}{2} e^{x^2}\right]_0^1 = 
		\frac{1}{2}\left(e - 1\right). 
	\end{equation*} 
	
  \textbf{Changement de variable.}

  On pose $u=x^2$, donc $du = 2x \, dx$. Les bornes de $x=0$ à $x=1$ deviennent des bornes de $u=0$ à $u=1$.
  Ainsi :
  $$\int  _{x=0} ^{x=1} e^{x^{2}}\, xdx
  = \int  _{u=0} ^{u=1} e^{u} \frac{du}{2}
  = \frac12\big[ e^u \big]_{u=0} ^{u=1}
  = \frac{1}{2}\left(e - 1\right)$$
	
	\item $\int  _{0} ^{\frac{\pi}{2}} \sqrt{3\cos(y)}\sin{(y)} \,dy $ 
	
	\textbf{Substitution.}
	On sait que $ \cos'(y) = -\sin (y) $. Alors $ \sqrt{3\cos(y)}\sin{(y)} $ 
	est de la forme $ -\sqrt{3} u'(y) f'(u(y)) $ avec $ u(y) = \cos(y) $ et 
	$ f(y) = \tfrac{2}{3} y^{3/2} $ et on obtient 
	\begin{equation*} 
		\int  _{0} ^{\frac{\pi}{2}} \sqrt{3\cos(y)}\sin{(y)} \,dy = 
		\left[-\sqrt{3} \, \frac{2}{3} \, (\cos(y))^{3/2}\right]_0^\frac{\pi}{2} = 
		-\frac{2}{\sqrt{3}} \left((\cos(\pi/2))^{3/2} - (\cos(0))^{3/2}\right) = 
		\frac{2}{\sqrt{3}}. 
	\end{equation*} 
	
	 \textbf{Changement de variable.}
	On pose $u=3\cos(y)$, donc $du = -3\sin(y) \, dy$. Les bornes de $y=0$ à $y=\frac\pi2$ deviennent des bornes de $u=3\cos(0)=3$ à $u=3\cos(\frac\pi2)=0$.
	$$\int  _{y=0} ^{y=\frac{\pi}{2}} \sqrt{3\cos(y)}\sin{(y)} \,dy
	= \int  _{u=3} ^{u=0}  \sqrt{u} \frac{-du}{3}
	= +\frac13 \int  _{u=0} ^{u=3}  \sqrt{u} \, du
	= \frac13 \left[ \frac23 u^{\frac32} \right]_{u=0} ^{u=3}
	= \frac13 \cdot \frac23 \cdot (\sqrt3)^3
	= \frac{2}{\sqrt3}$$
	
	\item $\int  _{0} ^{\frac{\pi}{4}} \tan(\theta) \,d\theta$
	
    \textbf{Substitution.}
	Comme $ \tan(\theta) = \frac{\sin(\theta)}{\cos(\theta)} $ et $ \cos'(\theta) 
	= -\sin(\theta) $, $ \tan(\theta) $ est de la forme $ u'(\theta) f'(u(\theta)) $ 
	avec $ u(\theta) = \cos(\theta) $ et $ f(\theta) = \ln(\theta) $ et on obtient 
	\begin{equation*} 
		\int  _{0} ^{\frac{\pi}{4}} \tan(\theta) \,d\theta = 
		\left[-\ln(\cos(\theta))\right]_0^\frac{\pi}{4} = 
		-\ln(\cos(\pi/4)) + \ln(\cos(0)) = 
		-\ln\left(\frac{1}{2} \sqrt{2}\right) = 
		\frac{1}{2} \ln(2). 
	\end{equation*} 
	
	\textbf{Changement de variable.} Poser $u=\cos(\theta)$.
	
	\item $\int  _{1} ^{5} \frac{e^{\sqrt{x}}}{2\sqrt{x}}dx$
	
    \textbf{Substitution.}
	Comme $ (\sqrt{x})' = \frac{1}{2 \sqrt{x}} $, $ \frac{e^{\sqrt{x}}}{2\sqrt{x}} $ 
	est de la forme $ u'(x) f'(u(x)) $ avec $ u(x) = \sqrt{x} $ et $ f(x) = e^{x} $ 
	et on obtient 
	\begin{equation*} 
		\int  _{1} ^{5} \frac{e^{\sqrt{x}}}{2\sqrt{x}}dx = 
		\left[e^{\sqrt{x}}\right]_1^5 = 
		e^{\sqrt{5}} - e. 
	\end{equation*} 
	
	\textbf{Changement de variable.} Poser $u=\sqrt{x}$.	
\end{enumerate} 
\fincorrection
\finexercice


% exercice 22
\exercice{}
\enonce
Calculer les intégrales suivantes~:
    \begin{examplescol}{4}
		\item $\int \limits _{0} ^{1} xe^{x}\,dx$ 
		\item $\int \limits _{0} ^{1} y^{2}e^{2y} \,dy$
		\item $\int \limits _{1} ^{2} \ln(t) \,dt$  
		\item $\int \limits _{0} ^{\pi} \theta \cos(\theta) \,d\theta$
    \end{examplescol}
\finenonce

\indication
Il s'agit d'appliquer la formule d'intégration par parties :
$$\int_a^b u(x) \, v'(x)\;dx= \big[uv\big]_a^b - \int_a^b u'(x) \, v(x)\;dx$$

\finindication

\correction

\video{C-D8-NNmh8g}

Nous allons appliquer la formule d'intégration par parties (IPP) :
$$\int_a^b u(x) \, v'(x)\;dx= \big[uv\big]_a^b - \int_a^b u'(x) \, v(x)\;dx$$
Pour cela nous allons dire quelle fonction joue le rôle de $u(x)$ et quelle fonction joue le rôle de $v'(x)$.


\begin{enumerate}
\item $\int  _{0} ^{1} xe^{x}\,dx$ 

On pose $u(x) = x$ et $v'(x) = e^x$.
Cela donne $u'(x) = 1$ et $v(x) = e^x$.
La formule d'intégration par parties donne :
$$I = \int  _{0} ^{1} xe^{x}\,dx = \big[xe^x\big]_0^1 - \int_0^1 1 \cdot e^x \;dx$$
Nous avons fait un progrès : le crochet est juste une évaluation à calculer et l'intégrale 
tout à droite est facile à calculer (car on connaît une primitive de $e^x$ qui est $e^x$). 
$$I = \left( 1\cdot e^1 - 0\cdot e^0\right) - \big[e^x\big]_0^1  = e - (e-1) = 1.$$

\item $\int  _{0} ^{1} y^{2}e^{2y} \,dy$

On pose $u(y) = y^2$ et $v'(y) = e^{2y}$.
Cela donne $u'(y) = 2y$ et $v(y) = \frac12e^{2y}$.
La formule d'intégration par parties donne :
$$
J = \int  _{0} ^{1} y^{2}e^{2y} \,dy \\
  = \big[\frac12 y^2 e^{2y}\big]_0^1 - \int  _{0} ^{1} ye^{2y} \,dy \\
  = \frac12 e^2 - \int  _{0} ^{1} ye^{2y} \,dy \\
$$

Ce n'est pas encore fini car il reste encore à calculer l'intégrale 
$J' = \int  _{0} ^{1} ye^{2y} \,dy$.
Mais on a quand même progresser car $J'$ est plus facile à calculer que $J$.

Pour calculer $J'$ on effectue une seconde intégration par parties, en posant 
$u(y) = y$ et $v'(y) = e^{2y}$ (et donc $u'(y) = 1$ et $v(y) = \frac12e^{2y}$).
Donc 
\begin{align*}
J' 
  &= \int  _{0} ^{1} ye^{2y} \,dy \\
  &= \left[\frac12ye^{2y}\right]_0^1 - \int_0^1 1 \cdot \frac12e^{2y} \;dy \\
  &= \frac12e^2 - \big[\frac14e^{2y}\big]_0^1 \\
  &= \frac12e^2 - \frac14(e^2-1) \\
  &= \frac14(e^2 + 1) \\
\end{align*}

Il ne reste plus qu'à reporter la valeur de $J'$ obtenue pour obtenir la valeur de $J$ :
$$J =  \frac12 e^2 - J' =  \frac12 e^2 - \frac14(e^2 + 1) =  \frac14(e^2 - 1)$$



\item $\int  _{1} ^{2} \ln(t) \,dt$  

Comment peut-on décider qui est $u$ et qui est $v$ alors qu'il n'y a pas de produit ?
Il suffit d'écrire $\ln(t) = \ln(t) \times 1$ pour faire apparaître artificiellement une multiplication.
On pose alors $u(t) = \ln(t)$ et $v'(t) = 1$ et donc $u'(t) = \frac1t$ et $v(t) = t$.
On peut maintenant faire une IPP :
\begin{align*}
\int  _{1} ^{2} \ln(t) \,dt
 &= \big[t\ln(t)\big] _{1} ^{2} - \int _{1} ^{2} \frac1t \cdot t \, dt \\
 &= (2\ln2-0) - \int _{1} ^{2} 1 \, dt \\
 &= 2\ln2 - \big[t\big] _{1} ^{2} \qquad \text{car une primitive de la fonction $1$ est $t$} \\ 
 &= 2\ln(2) - (2-1) \\
 &= 2\ln(2) - 1 \\
\end{align*}

\item $\int  _{0} ^{\pi} \theta \cos(\theta) \,d\theta$  

On pose $u(\theta) = \theta$ et $v'(\theta) = \cos(\theta)$.
Cela donne $u'(\theta) = 1$ et $v(\theta) = \sin(\theta)$.
La formule d'intégration par parties donne :
\begin{align*}
\int  _{0} ^{\pi} \theta \cos(\theta) \,d\theta 
 &= \big[\theta\sin(\theta) \big]_{0} ^{\pi} - \int_{0} ^{\pi} \sin(\theta) \,d\theta \\
 &= 0 - \big[ -\cos(\theta) \big]_{0} ^{\pi} \\
 &= \cos\pi - \cos0 \\
 &= -1-1 \\
 &= -2
\end{align*} 

\end{enumerate}
\fincorrection
\finexercice




% exercice 23
\exercice{}
\enonce\ 
\begin{enumerate}
	\item Déterminer $A$ et $B$ tels que $\frac{1}{x^{2}-1} = \frac{A}{x-1}+\frac{B}{x+1}$, puis calculer $\int \limits _{2} ^{4} \frac{1}{x^{2}-1} dx$.
	\item Déterminer $A, B, C$ tels que $\frac{x-2}{(x^{2}+1)(2x+1)} = \frac{Ax+B}{x^{2}+1}+\frac{C}{2x+1}$, puis calculer
	$\int \limits _{0} ^{1} \frac{x-2}{(x^{2}+1)(2x+1)}  dx$.
\end{enumerate}
\finenonce

\indication
Une primitive de $ \frac{u'(x)}{u(x)} $ est $ \ln|u(x)| + C $, $ C \in \Rr $. 
\finindication

\correction

\video{5ORyDi2LXnw}

\sauteligne
\begin{enumerate} 
	\item 
	On écrit 
	\begin{equation*} 
		\frac{A}{x-1}+\frac{B}{x+1} = \frac{A(x+1)+B(x-1)}{(x+1)(x-1)} = 
		\frac{Ax + A + Bx - B}{x^2 - 1} = \frac{x(A+B) + A - B}{x^2 - 1}. 
	\end{equation*} 
	Donc $ \frac{1}{x^{2}-1} = \frac{(A+B)x + A - B}{x^2 - 1} $ d'où 
	$ 1 = (A+B)x + A - B $ et on obtient le système suivant pour déterminer 
	les valeurs de $ A $ et $ B $~: 
	\begin{equation*} 
		\begin{cases} 
			A + B &= 0 \\ 
			A - B &= 1 
		\end{cases} 
		\quad \Longleftrightarrow \quad 
		\begin{cases} 
			A  &= -B \\ 
			2A &= 1 
		\end{cases} 
		\quad \Longleftrightarrow \quad 
		\begin{cases} 
			A &= \frac{1}{2} \\ 
			B &= -\frac{1}{2}. 
		\end{cases} 
	\end{equation*} 
	Ainsi $ \frac{1}{x^{2}-1} = \frac{1}{2(x-1)}-\frac{1}{2(x+1)} $ et donc : 
	\begin{equation*} 
		\begin{split} 
			\int \limits _{2} ^{4} \frac{1}{x^{2}-1} dx 
			&= 
			\int \limits _{2} ^{4} \left(\frac{1}{2(x-1)}-\frac{1}{2(x+1)}\right) dx 
			= 
			\int \limits _{2} ^{4} \frac{1}{2(x-1)} dx 
			- \int \limits _{2} ^{4} \frac{1}{2(x+1)} dx \\ 
			&= 
			\frac{1}{2} \int \limits _{2} ^{4} \frac{1}{x-1} dx 
			- \frac{1}{2} \int \limits _{2} ^{4} \frac{1}{x+1} dx 
			= 
			\frac{1}{2} \big[\ln|x-1|\big]_2^4 
			-\frac{1}{2} \big[\ln|x+1|\big]_2^4 \\ 
			&= 
			\frac{1}{2} \left(\ln(3) - \ln(1)\right)  
			-\frac{1}{2} \left(\ln(5) - \ln(3)\right) 
			= 
			\ln(3) - \frac{1}{2} \ln(5). 
		\end{split} 
	\end{equation*} 
	\item 
	On écrit 
	\begin{equation*} 
		\begin{split} 
			\frac{Ax+B}{x^{2}+1}+\frac{C}{2x+1} 
			&= \frac{(Ax+B)(2x+1) + C(x^2+1)}{(x^2+1)(2x+1)} 
			= \frac{2Ax^2 + Ax + 2Bx + B+ Cx^2+C}{(x^2+1)(2x+1)} \\ 
			&= 
			\frac{(2A+C) x^2 + (A + 2B)x + B + C}{(x^2+1)(2x+1)}. 
		\end{split} 
	\end{equation*} 
	Donc $ \frac{x-2}{(x^{2}+1)(2x+1)} = \frac{(2A+C) x^2 + (A + 2B)x + B 
		+ C}{(x^2+1)(2x+1)} $ d'où $ x - 2 = (2A+C) x^2 + (A + 2B)x + B + C $ 
	et on obtient le système suivant pour déterminer les valeurs de $ A, 
	B $ et $ C $~: 
	\begin{equation*} 
		\begin{cases} 
			2A + C &= 0 \\ 
			A + 2B &= 1 \\ 
			B + C  &= -2 
		\end{cases} 
		\quad \Longleftrightarrow \quad 
		\begin{cases} 
			2 - 4B - 2 - B &= 0 \\ 
			A &= 1 - 2B \\ 
			C &= -2 - B
		\end{cases} 
		\quad \Longleftrightarrow \quad 
		\begin{cases} 
			B &= 0 \\ 
			A &= 1 \\ 
			C &= -2 
		\end{cases} 
	\end{equation*} 
	Ainsi $ \frac{x-2}{(x^{2}+1)(2x+1)} = \frac{x}{x^{2}+1} - \frac{2}{2x+1} $ et donc : 
	\begin{equation*} 
		\begin{split} 
			\int \limits _{0} ^{1} \frac{x-2}{(x^{2}+1)(2x+1)}  dx 
			&= 
			\int_0^1 \left(\frac{x}{x^{2}+1} - \frac{2}{2x+1}\right) dx 
			= 
			\int_0^1 \frac{x}{x^{2}+1} dx - \int_0^1 \frac{2}{2x+1} dx \\ 
			&= 
			\frac{1}{2} \int_0^1 \frac{2x}{x^{2}+1} dx - \int_0^1 \frac{2}{2x+1} dx 
			= 
			\frac{1}{2} \big[\ln|x^2+1|\big]_0^1 - \big[\ln|2x+1|\big]_0^1 \\ 
			&= 
			\frac{1}{2} \left(\ln(2) - \ln(1)\right) - \left(\ln(3) - \ln(1)\right) 
			= \frac12\ln(2) - \ln(3). 
		\end{split} 
	\end{equation*} 
\end{enumerate} 
\fincorrection
\finexercice

\end{document}