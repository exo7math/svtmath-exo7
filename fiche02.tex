\documentclass[11pt,class=report,crop=false]{standalone}
\usepackage{exo7sv}

\begin{document}

%%%%%%%%%%%%%%%%%%%%%%%%%%%%%%%%%%%%%%%%%%%%%%%%%%%%%%%%%%%%%%%%%%%%%%
%%%%%%%%%%%%%%%%%%%%%%%%%%%%%%%%%%%%%%%%%%%%%%%%%%%%%%%%%%%%%%%%%%%%%%

\entete{Université de Lille}{Mathématiques pour la SVT}


\titre{Fiche 2. \quad Dérivées} 

\encadre{
\emph{Savoir.}
\begin{itemize}[label=$\square$]
  \item Comprendre la définition de la dérivée en terme de limite.
  \item Connaître les formules de la dérivées d'une somme, d'un produit, d'un quotient.
  \item Connaître sur le bout de doigts les formules des dérivées usuelles.
\end{itemize}
\emph{Savoir-faire.}
\begin{itemize}[label=$\square$]
  \item Savoir calculer l'équation de la tangente au graphe d'une fonction.
  \item Savoir dériver les fonctions construites à partir de fonctions usuelles.
  \item En particulier savoir dériver les compositions de fonctions.
\end{itemize}
}

%%%%%%%%%%%%%%%%%%%%%%%%%%%%%%%%%%%%%%%%%%%%%%%%%%%
\subsection*{Définition}

Le \textbf{nombre dérivé} d'une fonction $f$ en $x_0$ est défini par une limite :
\mybox{$\displaystyle f'(x_0) = \lim_{x\to x_0} \frac{f(x)-f(x_0)}{x-x_0}$}

Une autre écriture de cette limite est :
$$f'(x_0) = \lim_{h \to 0} \frac{f(x_0+h)-f(x_0)}{h}.$$

Une autre notation pour $f'(x_0)$ est $\frac{\dd f}{\dd x}(x_0)$.

La \textbf{fonction dérivée} est $x \mapsto f'(x)$. 

%%%%%%%%%%%%%%%%%%%%%%%%%%%%%%%%%%%%%%%%%%%%%%%%%%%
\subsection*{Tangente}

La dérivée en $x_0$ est le coefficient directeur de la tangente au point $(x_0,f(x_0))$.

L'équation de cette tangente est :
\mybox{$y = (x-x_0) f'(x_0) + f(x_0)$}

\begin{center}
\begin{tikzpicture}[scale=2]

	\draw[->,>=latex, gray, very thin] (-0.5,0) -- (3.3,0);
	\draw[->,>=latex, gray, very thin] (0,-0.5) -- (0,2.8);

%	\draw[domain=-0.25:2.5,black,thick,smooth] plot (\x,{0.6+0.4*\x+0.6*cos(4*\x r)});

    \draw[domain=0:2.35, blue,very thick,smooth] plot (\x,{2-(\x-1)^2)});


   \def \x{0.7}
    \coordinate (A) at ({\x},{2-(\x-1)^2)});
%    \fill (A) circle (1.5pt) node[above] {$M_0$};

    \draw[myred,thick] (A)--+(1.5,{1.5*(2-2*\x)}) node[below]{$T$};
    \draw[myred,thick] (A)--+(-1.5,{-1.5*(2-2*\x)}) ;

  \draw[dashed] (A -| 0,0) node[left]{$f(x_0)$} -- (A)--({\x},0) node[below]{$x_0$};

%\foreach \i in {4,3,...,1}
%{
%  \def\xx{\x + 1.5-0.3*\i};
%    \coordinate (M) at ({\xx},{2-(\xx-1)^2)});
%    \fill (M) circle (1.5pt);
%    \draw (A)--(M)--+($\i*(M)-\i*(A)$)--(A)--+($\i*(A)-\i*(M)$);
%    \coordinate (P) at ({\xx},0);
%};
%  \draw[dashed] (M)--(P) node[below]{$x$};
%  \node[above right] at (M) {$M$};

\end{tikzpicture}
\end{center}

Exemple : quelle est l'équation de la tangente au graphe de $f(x)=e^{2x}$ en $x_0=1$ ?
On a $f'(x) = 2e^{2x}$, $f(x_0)=f(1)=e^2$, $f'(x_0)=f'(1) = 2e^2$.
L'équation de la tangente est $y = (x-1)2e^2 + e^2$, ce qui s'écrit aussi
$y = 2e^2x - e^2$.

%%%%%%%%%%%%%%%%%%%%%%%%%%%%%%%%%%%%%%%%%%%%%%%%%%%
\subsection*{Opérations sur les dérivées}

\begin{itemize}
  \item \textbf{Somme.} \myboxinline{$(f+g)'=f'+g'$}

  \item \textbf{Produit par un réel.} \myboxinline{$(k f)' = k f' \qquad \text{ où } k\in \Rr$} 

  \item \textbf{Produit.} \myboxinline{$(f \times g)' = f'g+fg'$}

  \item \textbf{Inverse.} \myboxinline{$\displaystyle\left(\frac{1}{f}\right)'=-\frac{f'}{f^2}$}

  \item \textbf{Quotient.} \myboxinline{$\displaystyle \left(\frac{f}{g}\right)'=\frac{f'g-fg'}{g^2}$}
\end{itemize}

\emph{Exemple.} Calcul de la dérivée de $f(x) = xe^x + \ln(x)$.
Il y a un produit ($xe^x$) et une somme.
Ainsi :
\begin{align*}
f'(x) &= (xe^x)' + (\ln(x))' \quad \text{(somme)} \\
      &= (x)'e^x +  x(e^x)' + \frac1x \quad \text{(produit et dérivée de ln)} \\
      &= e^x +xe^x +\frac1x \quad \text{(dérivée de exp)} \\
      &= (x+1)e^x + \frac1x \\
\end{align*}


\emph{Exemple.} Calcul de la dérivée de $f(x) = \tan(x)$.
Par définition $\tan(x) = \frac{\sin(x)}{\cos(x)}$. Il s'agit de dériver un quotient.

\begin{align*}
\tan'(x) &= \left(\frac{\sin(x)}{\cos(x)}\right)'  \\
      &= \frac{\cos(x)\cos(x)-\sin(x)(-\sin(x))}{(\cos(x))^2} \\
      &= \frac{\cos^2(x) + \sin^2(x)}{\cos^2(x)} \\
      &= \frac{1}{\cos^2(x)} \\
\end{align*}
On a utilisé l'égalité $\cos^2(x)+\sin^2(x) = 1$.
En repartant de l'avant-dernière égalité on a aussi :
$$ \tan'(x) 
= \frac{\cos^2(x) + \sin^2(x)}{\cos^2(x)} 
= \frac{\cos^2(x)}{\cos^2(x)}+\frac{\sin^2(x)}{\cos^2(x)}
= 1 + \tan^2(x)$$

%%%%%%%%%%%%%%%%%%%%%%%%%%%%%%%%%%%%%%%%%%%%%%%%%%%
\subsection*{Formules}

Les dérivées des fonctions classiques (à gauche) et les formules pour la composition (à droite, où $u$ est une fonction qui dépend de $x$).

\begin{center}
%\noindent
\setlength{\arrayrulewidth}{0.05mm}
%\begin{tabular}{|l|l|l|} \hline
\begin{tabular}[t]{|c|c@{\vrule depth 1.2ex height 3ex width 0mm \ }|}
\hline
\textbf{Fonction}         & \textbf{Dérivée} \\ \hline
   $x^n$         & $nx^{n-1}$  \quad ($n \in \Zz$)   \\ \hline
   $\frac 1x$    & $-\frac{1}{x^2}$              \\ \hline
   $\sqrt{x}$    & $\frac12 \frac1{\sqrt{x}}$   \\ \hline
   $x^\alpha$   & $\alpha x^{\alpha-1}$  \quad ($\alpha\in\Rr$)  \\ \hline
   $e^x$         & $e^x$                        \\ \hline
   $\ln x$       & $\frac 1x$                   \\ \hline
   $\cos x$      & $-\sin x$                    \\ \hline
   $\sin x$      & $\cos x$                     \\ \hline
   $\tan x$      & $1+\tan^2 x = \frac{1}{\cos^2 x}$        \\ \hline
\end{tabular}
\hspace*{2cm}
\begin{tabular}[t]{|c|c@{\vrule depth 1.2ex height 3ex width 0mm \ }|}
\hline
\textbf{Fonction}         & \textbf{Dérivée} \\ \hline
   $u^n$         & $nu'u^{n-1}$  \quad  ($n \in \Zz$)   \\ \hline
   $\frac 1u$    & $-\frac{u'}{u^2}$              \\ \hline
   $\sqrt{u}$    & $\frac12 \frac{u'}{\sqrt{u}}$   \\ \hline
   $u^\alpha$   & $\alpha u' u^{\alpha-1}$ \quad ($\alpha\in\Rr$)  \\ \hline
   $e^u$         & $u'e^u$                        \\ \hline
   $\ln u$       & $\frac {u'}{u}$                   \\ \hline
   $\cos u$      & $-u'\sin u$                    \\ \hline
   $\sin u$      & $u'\cos u$                     \\ \hline
   $\tan u$      & $u'(1+\tan^2 u) = \frac{u'}{\cos^2 u}$        \\ \hline
\end{tabular}
\hfill
\end{center}

\emph{Exemples.}
\begin{itemize}
  \item $F(x) = \ln(x^2)$ alors en posant $u=x^2$ (et donc $u'=2x$), on a 
$F(x) = \ln(u)$ et donc $F'(x) = \frac{u'}{u} = \frac{2x}{x^2} = \frac2x$.

  \item $F(x) = \exp(\frac1x)$ alors en posant $u=\frac1x$ (et donc $u'=-\frac1{x^2}$), on a $F(x) = \exp(u)$ et donc
$F'(x) = u' \exp(u) = -\frac1{x^2}\exp(\frac1x)$ .

  \item $F(x) = \sqrt{\ln(x)}$ alors en posant $u=\ln(x)$ (et donc $u'=\frac1x$), on a $F(x) = \sqrt{u}$ et donc
$F'(x) = \frac12 \frac{u'}{\sqrt{u}} = \frac12\frac1{x}\frac{1}{\sqrt{\ln(x)}}$.
\end{itemize}


\end{document}