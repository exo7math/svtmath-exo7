
\input{preamb-pres.tex}
\usecolortheme[RGB={0,45,179}]{structure}




%%%%%%%%%%%%%%%%%%%%%%%%%%%%%%%%%%%%%%%%%%%%%%%%%%%%%%%%%%%%%
%%%%%%%%%%%%%%%%%%%%%%%%%%%%%%%%%%%%%%%%%%%%%%%%%%%%%%%%%%%%%



\begin{document}

% Color for video
\setbeamertemplate{background canvas}[vertical shading][top=couleurhaut,middle=couleurmilieu,midpoint=0.4,bottom=couleurbas] 

% White for print
%\setbeamertemplate{background canvas}[vertical shading][top=white,middle=white,midpoint=0.4,bottom=white] 


\begin{frame}

\thispagestyle{empty}    

\vfil

\Large

%\hspace*{-50ex}
\center
\begin{minipage}{0.8\textwidth}
\center
\enonce
    Déterminer le domaine de définition maximal des expressions suivantes :

    \begin{examplescol}{2}
        \item $f_1(x) = x^2 +x+1$
        \item $f_2(x) = \sqrt {x-1}$
        \item $f_3(x) = \sqrt{\frac{2 + 3 x}{5-2x}}$
        \item $f_4(x) = \ln (4 x + 3)$
    \end{examplescol}
\finenonce
\end{minipage}






\end{frame}


\end{document}