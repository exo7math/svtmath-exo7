\documentclass[11pt,class=report,crop=false]{standalone}
\usepackage{exo7sv}

\begin{document}

%%%%%%%%%%%%%%%%%%%%%%%%%%%%%%%%%%%%%%%%%%%%%%%%%%%%%%%%%%%%%%%%%%%%%%
%%%%%%%%%%%%%%%%%%%%%%%%%%%%%%%%%%%%%%%%%%%%%%%%%%%%%%%%%%%%%%%%%%%%%%

%%%%%%%%%%%%%%%%%%%%%%%%%%%%%%%%%%%%%%%%%%%%%%%%%%%%%%%%%%%%
\section*{\'Equations différentielles}


\setcounterexo{11}

\exercice{}
\enonce
On observe une population de microbes se développant de manière malthusienne, c'est-à-dire 
dont le taux de croissance au temps $t$ (en heures) est proportionnel à la taille de la
population $N(t)$.
    \begin{enumerate}
        \item
        En notant $k$ la constante de proportionnalité, donner une équation différentielle 
        modélisant cette situation.
        \item
        Montrer que les fonctions $N(t) = Ce^{kt}$ (où $C$ est une constante) sont solutions 
        de cette équation différentielle.
        \item
        Que représente $C$ ?
        \item
        Si $k= \ln 2$, que peut-on dire de la population au bout d'une heure ?
        \item
        Si $k = \ln 2$ et $N(0)=100$, quelle sera, d'après le modèle, la taille de la 
        population au bout de 4 heures et 20 minutes ?
        \item
        Si $k = \ln 2$ et $N(0)=100$, au bout de combien de temps la population 
        atteindra-t-elle 1000 individus ?
    \end{enumerate}  
\finenonce

\indication
L'équation différentielle est une égalité qui relie le taux de croissance $N'(t)$ à la population $N(t)$.
\finindication

\correction

\video{pMhAf48IjW8}

\sauteligne 
\begin{enumerate} 
\item 
Si $ N(t) $ d\'enote la taille de la population \`a l'instant $ t $, le taux de 
croissance au temps $ t $ est donn\'e par $ N'(t) $. Alors on obtient l'\'equation 
diff\'erentielle 
\begin{equation} \label{eq:n1} 
N'(t) = k N(t). 
\end{equation} 
\item 
Soit $ N(t) = C e^{kt} $. Sa d\'eriv\'ee est 
\begin{equation*} 
N'(t) = (C e^{kt})' = C(e^{kt})' = Cke^{kt}. 
\end{equation*} 
De plus, $ k N(t) = kCe^{kt} = Cke^{kt} $. Comme on obtient $N'(t) = kN(t)$, alors les fonctions de la forme $ N(t) = C e^{kt} $, $ C \in \Rr $, sont 
solutions de (\ref{eq:n1}). 
\item 
Comme $ N(0) = C e^0 = C $, $ C $ est la taille de la population au d\'ebut. 
Donc $ N(t) = N(0) e^{kt} $. 
\item 
On remplace $ k $ par $ \ln(2) $. Alors $ N(1) = N(0) e^{\ln(2)} = 2 N(0) $ car $e^{\ln(2)}=2$. 
Donc au bout d'une heure la population a doubl\'e. 
\item 
Pour $ N(0) = 100 $ et $ k = \ln(2) $ on a $ N(t) = 100\cdot e^{t\ln(2)} = 100 \cdot (e^{\ln (2)})^t = 100 \cdot 2^t $. 
Donc 
\begin{equation*} 
N\left(4+\frac{1}{3}\right) = N\left(\frac{13}{3}\right) = 100 \cdot 
2^{13/3} \approx 2015. 
\end{equation*} 
\item 
On a $ N(t) = 100 \cdot 2^t $ et on cherche le temps $ T $ tel que 
$ N(T) = 100 \cdot 2^T = 1000 $. 
\begin{equation*} 
100 \cdot 2^T = 1000  \Longleftrightarrow  2^T = 10 
\Longleftrightarrow \ln(2^T) = \ln(10) 
 \Longleftrightarrow T \ln(2) = \ln(10)  
 \Longleftrightarrow T = \frac{\ln(10)}{\ln(2)} \approx 3,32. 
\end{equation*} 
Au bout de 3 heures et 20 minutes environ la population atteindra 1000 individus. 
\end{enumerate} 
\fincorrection 
\finexercice


\setcounterexo{14}
\exercice{}
\enonce
On considère le problème :
\begin{align}
  y'(t) &= -t^2 y(t) \quad \text{pour $t \geq 0$,} \label{eq}\\
  y(0) &= y_0. \label{ci}
\end{align}
\begin{enumerate}
  \item Montrer que la fonction constante $y(t)=0$ est solution de \eqref{eq}.
  Y a-t-il d'autres solutions constantes?
  \item On suppose que $y$ est une solution de \eqref{eq} et \eqref{ci} et on suppose $y_0>0$.
  \begin{enumerate}
    \item Montrer que $y(t)>0$ pour tout $t\ge 0$ ({\it on admettra que les graphes de deux solutions distinctes de (\ref{eq}) ne se coupent jamais}).
    \item Montrer que $y$ est décroissante sur $[0; +\infty[$.
    En déduire que $y(t)$ admet une limite quand $t \to +\infty$.
  \end{enumerate}
  \item On pose $u(t)=at^3+b$ et $Y(t)=e^{u(t)}$ où $a$ et $b$ sont des constantes.
  \begin{enumerate}
    \item Calculer $u'(t)$ et $Y'(t)$.
    \item Déterminer la constante $a$ pour que $Y(t)$ soit solution de (\ref{eq}).
    \item Calculer $Y(0)$.
    Comment faut-il choisir $b$ pour avoir $Y(0)=y_0$?
    \item Calculer $\lim_{t\to +\infty}Y(t)$, en prenant pour $a$ et $b$ les valeurs trouvées dans les questions précédentes.
  \end{enumerate}
\end{enumerate}
\finenonce

\indication
Pour a), remplacer $y$ dans \eqref{eq} par une fonction constante.
Pour b) (ii) utiliser l'équation \eqref{eq}.
\finindication

\correction

\video{RXIw8BenXV4}

\sauteligne
\begin{enumerate}
  \item Soit $c \in \mathbb{R}$.
  Soit $y(t)=c$ la fonction constante égale à $c$ sur $[0; +\infty[$.
  On a $y'(t)=0$ pour tout $t \in [0; +\infty[$.
  Donc $y$ est solution constante de \eqref{eq} si et seulement si :
  \begin{equation*}
  0 = -t^2 c \quad \text{pour tout $t \in [0; +\infty[$.}
  \end{equation*}
  Cela n'est vrai que si $c=0$.
  Ainsi la fonction constante égale à $0$ est bien solution de \eqref{eq} et aucune autre fonction constante n'est solution de \eqref{eq}.
  
  \item Soit $y$ une solution de \eqref{eq} et \eqref{ci} avec $y_0>0$.
  \begin{enumerate}
    \item Comme $y_0>0$, $y$ n'est pas la fonction constante égale à $0$.
    Ainsi le graphe de $y$ ne coupe jamais le graphe de la fonction constante égale à $0$, c'est-à-dire l'axe des abscisse.
    Comme $y$ est une fonction continue, on en déduit qu'elle est de signe constant.
    Comme $y_0>0$, on en conclut que $y(t)>0$ pour tout $t \in [0; +\infty[$.
    \item En utilisant l'équation \eqref{eq}, on déduit de la question précédente que $y'(t)\le0$ pour tout $t \in [0; +\infty[$.
    Donc $y$ est décroissante sur $[0; +\infty[$.
    Comme est elle minorée (par $0$), on en déduit qu'elle admet une limite ($l \geq 0$) quand $t \to +\infty$.
  \end{enumerate}
  \item
  \begin{enumerate}
    \item On calcule pour tout $t \in [0; +\infty[$:
    \begin{align*}
      u'(t) &= 3at^2, \\
      Y'(t) &= u'(t) e^{u(t)} = 3at^2 e^{at^3+b}.
    \end{align*}
    \item $Y$ est solution de \eqref{eq} si et seulement si :
    \begin{equation*}
    3at^2 e^{at^3+b} = -t^2 e^{at^3+b} \quad \text{pour tout $t \in [0; +\infty[$.}
    \end{equation*}
    Cela est vrai si et seulement si $a=-\frac13$.
    \item On a $Y(0) = e^b$.
    On en déduit que $Y(0) = y_0$ si et seulement si $b=\ln(y_0)$.
    \item Avec $a=-\frac13$ et $b=\ln(y_0)$ on a
    \begin{equation*}
    Y(t) = y_0e^{-\frac{t^3}3} \quad \text{pour tout $t \in [0; +\infty[$.}
    \end{equation*}
    On en déduit que $\lim_{t \to +\infty} Y(t) = 0$.
  \end{enumerate}
\end{enumerate}
\fincorrection
\finexercice


\setcounterexo{17}

\exercice{}
\enonce
On considère le problème représenté par les deux équations suivantes:
            \begin{eqnarray}
                y'(t) & = & (y(t)-1)^2(y(t)+1)\label{eqpoly}\\
                y(0) & = & 0 \label{cipoly}
            \end{eqnarray}
            et l'on suppose qu'il existe une fonction $ y:\Rr^+\to\Rr$ solution de ce problème.
            \begin{enumerate}
                \item Chercher les solutions constantes de l'équation différentielle 
                      (\ref{eqpoly}).
                \item Montrer que la solution $y$ du problème est croissante sur $\Rr^+$.
                \item En déduire que $\displaystyle \lim_{t\to +\infty}y(t)=l$ avec $l \in
                  \Rr $. Calculer $l$ en admettant que $\displaystyle \lim_{t\to +\infty}y'(t)=0$.
            \end{enumerate}
\finenonce

\indication
Vous devez étudier le signe de $ y'(t) $ en admettant que les graphes 
de deux solutions distinctes de (\ref{eqpoly}) ne se coupent jamais. 
\finindication

\correction

\video{QWLnhrKataE}

\sauteligne 
\begin{enumerate} 
\item 
On pose $ y(t) = c $, $ t \geq 0 $. Donc $ y'(t) = 0 $, $ t \geq 0 $. Si on remplace 
$ y'(t) $ par $ 0 $ et $ y(t) $ par $ c $ dans (\ref{eqpoly}) on obtient 
\begin{equation*} 
0 = (c - 1)^2 (c + 1) \quad \Longleftrightarrow \quad c-1 = 0 \ \mbox{ ou }\  c+1 = 0  
\quad \Longleftrightarrow \quad c = 1 \mbox{ ou } c = -1. 
\end{equation*} 
Donc les solutions constants sont $ y_1(t) = -1 $ et $ y_2(t) = 1 $, $ t \geq 0 $. 
\item 
\'Etudions le signe de $ y $. On sait que $ y(0) = 0 $ et donc $ y_1(0) < y(0) < 
y_2(0) $. Comme les graphes graphes de deux solutions distinctes de (\ref{eqpoly}) 
ne se coupent jamais et $ y_1(0) < y(0) < y_2(0) $, le graphe de $ y $ est toujours 
en dessous de celui de $ y_2 $ et au-dessus de celui de $ y_1 $. Alors 
\begin{equation*} 
-1 = y_1(t) < y(t) < y_2(t) = 1, \ t \geq 0, 
\end{equation*} 
et on obtient le tableau de variation suivant~: 
\begin{equation*} 
\begin{tabvar}{|C|CCC|} \hline 
t            & 0 && +\infty \\ \hline 
(y(t) - 1)^2 && + & \\ \hline 
y(t) + 1     && + & \\ \hline 
y'(t)        && + & \\ \hline 
\niveau{1}{2} y(t) & 0 & \croit & \\ \hline 
\end{tabvar} 
\end{equation*} 
Comme $y'(t)\ge0$, alors la fonction $y$ est croissante.
\item 
Comme la fonction $ y(t) $ est croissante et major\'ee par $ 1 $ (car $y(t) \le y_2(t) = 1$), la limite 
$ \displaystyle l := \lim_{t\to +\infty}y(t) $ existe. De plus, $ 0 \leq l 
\leq 1 $. En admettant que $ \displaystyle \lim_{t\to +\infty}y'(t) = 0 $ et 
en utilisant (\ref{eqpoly}), on obtient~: 
\begin{equation*} 
0 = \lim_{t \to +\infty} y'(t) = \lim_{t \to +\infty} (y(t)-1)^2(y(t)+1) = 
(\lim_{t \to +\infty} y(t) - 1)^2 (\lim_{t \to +\infty} y(t) + 1) = (l-1)^2 (l+1). 
\end{equation*} 
Comme $ 0 = (l-1)^2 (l+1) \Longleftrightarrow l = -1 $ ou $ l = 1 $, mais $ l 
\geq 0 $, on d\'eduit donc que $ l = 1 $. 
\end{enumerate} 
\fincorrection 
\finexercice





\end{document}